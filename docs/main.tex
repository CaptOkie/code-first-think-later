\documentclass[12pt,letterpaper]{article}

\usepackage[T1]{fontenc}
\usepackage[margin=0.75in,headheight=1.5em]{geometry}
\usepackage{enumitem}
\usepackage{fancyhdr}
\usepackage{lastpage}
\usepackage{float}
\usepackage{tabu}
\usepackage{booktabs}
\usepackage{graphicx}
\usepackage{lmodern}
\usepackage[table]{xcolor}  
\usepackage{transparent}

\begin{document}

\renewcommand\headrule{}

\pagestyle{fancy}
\fancyhf{}
\lfoot{COMP 3004}
\rfoot{\thepage/\pageref{LastPage}}
\cfoot{Requirements Analysis Document}

% CUSTOM COMMANDS
\newcommand{\teamname}{Code First, Think Later}
\newcommand{\personone}{Kevin Hua}
\newcommand{\persontwo}{Hendrik Knoetze}
\newcommand{\personthree}{Juhandr\'e Knoetze}
\newcommand{\myparagraph}[1]{\paragraph{#1}\mbox{}\\}

%% USE-CASE COMMANDS
\newcounter{usecasenum}
\renewcommand{\theusecasenum}{UC-\ifnum\value{usecasenum}<10 0\fi\arabic{usecasenum}}
\newcommand{\newusecase}[1]{\refstepcounter{usecasenum}\label{#1}\expandafter\newcommand\csname #1\endcsname{\ref{#1}}}

\newusecase{participateinprojects}
\newusecase{manageprojects}
\newusecase{createnewproject}
\newusecase{launchppid}
\newusecase{viewppidresults}
\newusecase{viewppidsummary}
\newusecase{viewppiddetails}
\newusecase{editproject}
\newusecase{editprojectdetails}
\newusecase{editteamsize}
\newusecase{editprojectname}
\newusecase{addstudent}
\newusecase{removestudent}
\newusecase{joinproject}
\newusecase{leaveproject}
\newusecase{editprofile}
\newusecase{editpersonalvalues}
\newusecase{editdesiredvalues}
\newusecase{saveerror}
\newusecase{openerror}
\newusecase{readerror}
\newusecase{writeerror}
\newusecase{closeerror}
\newusecase{invalidinputerror}
\newusecase{illegalminerror}
\newusecase{illegalmaxerror}
\newusecase{projectexistserror}
\newusecase{invalidstudenterror}
\newusecase{insufficientstudentserror}
\newusecase{invalidprojecterror}
%% END USE-CASE COMMANDS

% TABLE STYLING
\everyrow{\hline}
\tabulinesep=0.5em
\setlength\extrarowheight{0.5em}
% END TABLE STYLING

%% TABLE COMMANDS
\definecolor{thcolor}{RGB}{193,193,193}
\newcommand{\ccindent}{\hspace{1.5em}\hangindent=1.5em}
\newcommand{\tableheader}{\rowfont\bf\rowcolor{thcolor!30}}
%% END TABLE COMMANDS
% END CUSTOM COMMANDS

\thispagestyle{empty}

\begin{center}
	CARLETON UNIVERSITY
\end{center}

\vfill

\begin{center}
	{\fontsize{55pt}{55pt}\selectfont cuPID}
	\vspace{0.5em}\rule{\textwidth}{0.5pt}
	Requirements Analysis Document
\end{center}

\vspace{5em}

\begin{center}
	\textbf{Team [\teamname{}]}\\
	\personone{}\\
	\persontwo{}\\
	\personthree{}
\end{center}

\vfill

\begin{center}
	Submitted to:\\
	Dr. Christine Laurendeau\\
	COMP 3004: Object Oriented Software Engineering\\
	School of Computer Science\\
	Carleton University
\end{center}

\vspace{2em}

\begin{center}
	\today
\end{center}

\newpage{}

\tableofcontents{}

\renewcommand{\listfigurename}{Figures}
\listoffigures

\renewcommand{\listtablename}{Tables}
\listoftables

\newpage{}

\section{Introduction}

\begin{center}
    -- project --
\end{center}

\begin{center}
	\Huge [cuPID]
\end{center}

\begin{center}
    \rule{0.85\textwidth}{0.5pt}
\end{center}

\subsection{Purpose of System}

Team projects are typically assigned in University courses in order to develop and foster
good teamwork related skills, which are crucial in most future endeavours, notably 
prospective employment. Unfortunately, the task of separating students into balanced 
and compatible teams has always been a nigh impossible feat - hardly a year passes that doesn't
boast at least a single team brimming with  contention. There just seems to be too many
nuances that are involved in building a perfect team for professors to account for, often leading
to random or pseudo-random assignment of teams. Another possibility that professors might employ
would be to allow the students themselves to form their teams. Regrettably, this option also
leads to much strife - students tend to select friends or acquaintances as partners. While this
solution might seem good, it is sadly the case that good friends often make bad project partners. 
There is also the case that many students have not made friends or acquaintances yet that they 
could ask to be their partners, thus leaving them to form a team with others in their situation.

Both current options leave much to be desired. Our firm, [\teamname{}], has been hired
to design a system that would provide a better solution to this long-standing issue. Thus, the purpose of
our system is to separate students into teams with others of similar personality and skills, eliminating 
the frequent torment associated with ill-matched teams.

\subsection{Overview of Document}

This report provides an overview of the design decisions we made for the cuPID project, including 
use cases, object models, and dynamic models. The goal of this document is to clearly impart our vision of this
program to Dr. Christine Laurendeau, our esteemed employer. In furtherance to this goal, we have painstakingly 
organized our report to maximize both basic legibility and traceability. Before delving into the more complex details 
of our proposed system, we have included a general overview of the elements we wish to incorporate, as well as 
some details with regards to our unique and innovative algorithm.

Following the short synopsis, you will find, the requested requirements, cleanly organized into separate functional 
and non-functional tables. Note that each requirement is assigned a unique but informative identifier - all future references
to this particular requirement will cite its corresponding code. The purpose of the system is, again, to ensure that the
reading of this document is as simple and pleasant an experience as possible. 

Directly after the requirements, we have the various system models. We have chosen to separate them into three
primary categories: the use cases, the objects, and the dynamic models. The goal of this separation scheme is to
stagger the introduction of different components such that the reader does not feel bombarded. Thus, you will 
first see the various possible scenarios and how the system will flow with regards to these scenarios. Subsequently, 
the various objects will be introduced and described in detail. Finally, we will delve into the more detailed sequence 
diagrams that will build on information from the prior sections. This section will handle such things as timings and
the flows for the various possible states of the system.

\vspace{1em}

\noindent Best regards,

\vspace{1em}

\begin{center}
	\includegraphics[scale=0.4]{imgs/logo.png} \\ \footnotesize{(Team [Code First, Think Later])}
\end{center}

\vspace{1em}

\section{Proposed System}

\subsection{Overview}

The system that we are proposing to build offers a simplistic and clean interface designed for the single purpose of generating 
well-rounded and personality-compatible teams. To this end, we will allow two possible users: the Administrator, typically the professor,
and the Student. We want to really make a clear distinction between these two users - all that the Student user can do is access their
personal profile and control their participation to projects, while the Administration is in charge of all other aspects of project maintenance. 
Solely the Administrator will be able to run and view the results of our specialized Partner Matching Algorithm. This algorithm, broadly speaking,
will be analyzing four aspects of a person in order to generate the best possible teams:  programming or other relevant skills, leadership skills,
personality, and available. 

In this section, we will showcase a comprehensive breakdown of all the various components of our proposed system. For the sake of 
legibility, we have decided to organize the content in such a way that we move naturally from simplest to most complex - in other words,
each subsequent section builds off the previous. 

\subsection{Functional Requirements}

Functional requirements are the result of us carefully analzing the problem statement that you have submitted and extracting all the key 
desired features. In other words, we have converted your request into a list of requirements that we will fulfill. For example, you mention that
it is essential that this software can create projects. All the main features that you have expressed the desire to have have been matched with
the appropriate user and included in the following table. This format has the added advantage of allowing you, our employer, to see for 
yourself that each and every requirement is carried out to your satisfaction. Note again the unique identifier preceding each of the functional 
requirements - this will allow you to more easily trace our solution to each. 

\begin{table}[H]
	\caption{Functional Requirements}
	\vspace{1em}
	\begin{tabu} to \textwidth {>{\bf}l X}
	    \tableheader{}ID & Description\\
		F-01 & Students must be able to add themselves to any number of projects. \\
		F-02 & Students must be able to remove themselves from a project they are currently part of. \\
		F-03 & Students must be able to edit their Project Partner Profile. \\
		\ccindent{}F-03-01 & \ccindent{}Students must be able to modify the values representing their own qualifications. \\
		\ccindent{}F-03-02 & \ccindent{}Students must be able to modify the values representing what qualifications they
		desire in potential partners (what they prioritize, for example). \\
		F-04 & Students must be able to view their Project Partner Profile. \\
		F-05 & Administrators must be able to create new projects. \\
		F-06 & Administrators must be able to edit existing projects. \\
		\ccindent{}F-06-01 & \ccindent{}Administrators must be able to modify the team size parameter for individual projects. \\
		\ccindent{}F-06-02 & \ccindent{}Administrators must be able to add a student to a selected project. \\
		\ccindent{}F-06-03 & \ccindent{}Administrators must be able to remove a student from a selected project. \\
		\ccindent{}F-06-04 & \ccindent{}Administrators must be able to modify the project name parameter for individual projects.\\
		F-07 & Administrators must be able to launch the PPID for a specific project. \\
		F-08 & Administrators must be able to view the summary results of the last-run PPID. \\
		F-09 & Administrators must be able to view the detailed results of the last-run PPID. \\
	\end{tabu}
\end{table}

\subsection{Non-Functional Requirements}

Non-functional requirements are differentiable from the functional requirements in that they are not the requested features of the project, but
rather the various standards our proposed system must meet. For example, you expressed the desire that the system be designed specifically 
for use with the Linux Ubuntu platform, and that all code must be written in the C++ programming language. Also included in this section are
assorted promises to meet various standards that you have not included in your problem statement but that either we believe should be required, 
or is required by law. Once again, each of these requirements have been assigned a unique identifier and can be used to trace our response to them.

\begin{table}[H]
	\caption{Non-Functional Requirements}
	\vspace{1em}
	\begin{tabu} to \textwidth {>{\bf}l >{\it}l X}
	    \tableheader{}ID & Category & Description\\
		NF-01 & Usability & The option to customize the key bindings for default keyboard shortcuts must be available for all users.\\
		NF-02 & Usability & The system will not terminate due to errors 95\% of the time.\\
		NF-03 & Implementation & The software must be written in the C++ language. \\
		NF-04 & Implementation & The software must use the Qt framework for creating the graphical user interface (G.U.I.). \\
		NF-05 & Implementation & The software must run in Linux environment. \\
		NF-06 & Performance & The PPID will take no longer than 5 seconds to finish running. \\
		NF-07 & Reliability & Saved data will remain uncorrupted at least 95\% of the time. \\
		NF-08 & Reliability & Saved information must be backed up. \\
		NF-09 & Supportability & The system must be extensible to a client/server architecture. \\
		NF-10 & Legal & Users must accept a Terms of Service agreement in order to use the software.\\
		NF-11 & Interface & The software must provide an API to cuLearn.\\
		NF-12 & Operations & A forum should be provided for users to report any bugs they may encounter. \\
		NF-13 & Packaging & The software will be available to download as a standalone executable. \\
	\end{tabu}
\end{table}

\subsection{System Models}

System models are some great tools that we have decided to incorporate into our system proposition to better convey the various functionalities and
interactions in our system. They provide a strong visual representation of the various ideas that we have, including general scenarios (use-cases) such as 
"Administrator creates a new project," interactions between different objects, and precise sequence diagrams that display what each object %if control object, add here
does and for approximately how long. Specifically, you will be provided with use case diagrams with corresponding descriptions, entity objects with corresponding  
dictionaries, state machine diagrams, and sequence diagrams. 

Use case diagrams and descriptions are useful for showcasing all the possible actions we want to incorporate into our system, with the added bonus of clearly
identifying the requirements each satisfies. Entity objects delve deeper and repesent all the users and other objects in the game, listing all the attributes and associations
that each has. The purpose of including this model is to exhibit all the potential behavior and aspects each object has access to. The next model, state machines, takes a large
step backwards; here we see a more abstracted view of the system - specifically, a diagram showing the possible states that each control object can experience. In other words, this
model does not see the system as a whole, but rather focuses on a very minute aspect of the system and explains how it works. Finally. sequence diagrams
expand on the previously mentioned use cases and transforms the detailed descriptions into visual representations that also display the proportion of time spent
at each step. Furthermore, the creation and destruction of objects is also displayed in this diagram. While it can be argued that each model exhibits varying levels
of informativeness, each is required to fully understand the next.

\subsubsection{Use Case Model}

Use case models are pairings of diagrams that depicts the actions that a specified user can take in a tree-like format with detailed descriptions of each action. The actions 
are divided into two three types: the high-level cases, denoted with <<initiates>>, the detailed cases, denoted with <<include>>, and the error cases, denoted with 
<<extend>>. To augment readability, we have decided to have a separate section for the high-level cases and grouping the detailed and error cases together. One 
concession we did make was to separate the different actions by user, or actor. 

The format of this section will be separated in two parts. For the first part, we will further subdivide into two parts: the high-level use cases and the detailed use cases. 
These two sub-sections will be identically formatted insofar as the diagrams will precede the more detailed descriptions, which are organized in a clean table format. Following
this first part, the second part will contain the more detailed flow of events for every use case in the previous part (both high-level and detailed).

\subsubsection*{High-level Use Cases}

This section contains the high-level use cases - essentially, the most abstracted action each user (actor) can take. The diagram below illustrates the interaction between
users, defined by the stick figure, and the system, defined by the rectangular box. The ovals represent the highest level actions that can be taken, with a connecting line to
the user that can initiate them. Specifically, we have the Student user who can initiate the high-level action of ParticipateInProjects, and we have the Administrator user who can
initiate in the action of ManageProjects. These actions are purposefully vague and general because they are the "parent action" that includes several other more specific actions.
\vspace{1em}

\begin{figure}[H]
	\centering{}
	\includegraphics[scale=0.3]{imgs/high-level-use-case-diagram.png}
	\caption{High-level Use Case Diagram}
\end{figure}

\begin{table}[H]
	\caption{High-Level Use Case Descriptions}
	\vspace{1em}
	\begin{tabu} to \textwidth {l >{\bf}l X}
	    \tableheader{}ID & Name & Description\\
		\participateinprojects{} & ParticipateInProjects & The student accesses and can modify their own profile details 
		(includes project participation details)\\
		\manageprojects{} & ManageProjects & The Administrator manages current project as well as having the option to 
		create a new instance of a project (includes manipulating student participation and access to the PPID and its results). \\
	\end{tabu}
\end{table}

\subsubsection*{Detailed Use Cases \& Error Use Cases}

\begin{figure}[H]
	\centering{}
	\includegraphics[scale=0.26]{imgs/detailed-administrator-use-case-diagram.png}
	\caption{Detailed Administrator Use Case Diagram}
\end{figure}

\begin{figure}[H]
	\centering{}
	\includegraphics[scale=0.3]{imgs/detailed-student-use-case-diagram.png}
	\caption{Detailed Student Use Case Diagram}
\end{figure}

\begin{figure}[H]
	\centering{}
	\includegraphics[scale=0.3]{imgs/detailed-user-input-use-case-diagram.png}
	\caption{Detailed User Input Use Case Diagram}
\end{figure}

\begin{figure}[H]
	\centering{}
	\includegraphics[scale=0.3]{imgs/detailed-common-error-use-case-diagram.png}
	\caption{Detailed Common Error Use Case Diagram}
\end{figure}

\begin{table}[H]
	\caption{Detailed Use Case Descriptions - Common Errors}
	\vspace{1em}
	\begin{tabu} to \textwidth {l >{\bf}l X}
	    \tableheader{}ID & Name & Description\\
		UC & StorageError &  The system reports that there is an error with the stored information for cuPID. \\
		UC & StorageReadError & The system reports that the selected file could not be read. \\
		UC & StorageOpenError & The system reports that the selected file could not be opened. \\
		UC & StorageWriteError & The system reports that the selected file could not be written to. \\
		UC & StorageCloseError & The system reports that the selected file could not be closed. \\
	\end{tabu}
\end{table}

\begin{table}[H]
	\caption{Detailed Use Case Descriptions - Administrators}
	\vspace{1em}
	\begin{tabu} to \textwidth {l >{\bf}l X}
	    \tableheader{}ID & Name & Description\\
		\createnewproject{} & CreateNewProject & The Administrator creates a new instance of a project.\\
		\launchppid{} & LaunchPPID & The Administrator runs the PPID for a specified project.\\
		\viewppidresults{} & ViewPPIDResults & The Administrator views the results of the PPID for a project. \\
		\viewppidsummary{} & ViewPPIDSummary & The Administrator views a summary of the groups that are created. \\
		\viewppiddetails{} & ViewPPIDDetails & The Administrator views a detailed analysis of the grouping decisions. \\
		\editproject{} & EditProject & The Administrator modifies an aspect of the project (includes student participation 
		and project parameters).\\
		\editprojectdetails{} & EditProjectDetails & The Administrator modifies the parameters of a specified project.\\
		\editteamsize{} & EditTeamSize & The Administrator sets the minimum and maximum team sizes to be used by the PPID. \\
		\editprojectname{} & EditProjectName & The Administrator sets the name of the project. \\
		\addstudent{} & AddStudent & The Administrator adds a Student to the current project.\\
		\removestudent{} & RemoveStudent & The Administrator removes a Student from the current project.\\
	\end{tabu}
\end{table}

\begin{table}[H]
	\caption{Detailed Use Case Descriptions - Students}
	\vspace{1em}
	\begin{tabu} to \textwidth {l >{\bf}l X}
	    \tableheader{}ID & Name & Description\\
		\joinproject{} & JoinProject & The student adds themselves to an existing project.\\
		\leaveproject{} & LeaveProject & The student removes themselves from an existing project.\\
		\editprofile{} & EditProfile & The student modifies the parameters of their profile (includes EditPersonalValues and EditDesiredValues).\\
		\editpersonalvalues{} & EditPersonalValues & The student modifies the personal details in his profile (what they have to offer). \\
		\editdesiredvalues{} & EditDesiredValues & The student modifies the desired details in his profile (what they are looking for in potential team members). \\
	\end{tabu}
\end{table}

\begin{table}[H]
	\caption{Detailed Use Case Descriptions - Error Cases}
	\vspace{1em}
	\begin{tabu} to \textwidth {l >{\bf}l X}
	    \tableheader{}ID & Name & Description\\
		\saveerror{} & SaveError & There was an error during saving (generalizes WriteError, ReadError, OpenError, CloseError). 
		The system reports the specific error to the user and aborts the operation.\\
		\openerror{} & OpenError & The system reports that the selected file could not be opened.\\
		\readerror{} & ReadError & The system reports that the selected file could not be read.\\
		\writeerror{} & WriteError & The system reports that the selected file could not be written to.\\
		\closeerror{} & CloseError & The system reports that the selected file could not be closed.\\
		\invalidinputerror{} & InvalidInputError & The system reports that the given value is invalid.\\
		\illegalminerror{} & IllegalMinError & The system reports that the given minimum team size is illegal.\\
		\illegalmaxerror{} & IllegalMaxError & The system reports that the given maximum team size is illegal.\\
		\projectexistserror{} & ProjectExistsError & The system reports that a project with the given name already exists.\\
		\invalidstudenterror{} & InvalidStudentError & The system reports that an invalid student, be it no student or a non-existent student, 
		was selected to carry out the operation.\\
		\insufficientstudentserror{} & InsufficientStudentsError & The system reports that there are not enough students registered in project to
		 run the PPID.\\
   		\invalidprojecterror{} & InvalidProjectError & The system reports that an invalid project, be it no project or a non-existent project, was selected to carry out the operation.\\
	\end{tabu}
\end{table}

\subsubsection*{Use Case Flow of Events}

\begin{center}
	\begin{tabu} to \textwidth {>{\it}l X}
		\toprule
		Use Case Identifier & \participateinprojects{} \\
		Name & {\bf ParticipateInProjects} \\
		Participating Actors & Initiated by Student \\

		Flow of Events & 
	    \begin{enumerate}[topsep=-1em,leftmargin=*]
		    \item[1.] The Student begins the cuPID process.
			\begin{enumerate}
				\item[2.] The Student is presented with a menu from the system. Options of: joining a project, leaving a project, or editing their profile for a project.
				\item[3.] If the Student chooses to edit their profile, the system brings up a menu of profile elements that may be modified. The Student may choose to edit their personal details (include use case \textbf{EditPersonalValues}) or to edit their desired partner values (include use case \textbf{EditDesiredValues}).
				\item[4.] If the Student chooses to join a project, the system presents a list of projects to join (include use case \textbf{JoinProject}).
				\item[5.] If the Student chooses to leave a project, the system presents a list of their joined projects to leave (include use case \textbf{LeaveProject}).
			\end{enumerate}
		\end{enumerate} \\

		Entry Conditions & \\

		Exit Conditions & \\ % KEVIN_QUESTION Do we need an exit condition?

		Quality Requirements &
		\begin{itemize}[topsep=-1em,leftmargin=*]
			\item The user must be operating cuPID as a Student.
		\end{itemize} \\

		Traceability & F-01, F-02, F-03, F-04\\

		\toprule
	\end{tabu}
\end{center}

\begin{center}
    \begin{tabu} to \textwidth {>{\it}l X}
        \toprule
		Use Case Identifier & \manageprojects{} \\
		Name & {\bf ManageProjects} \\
        Participating Actors & Initiated by Administrator \\
		Flow of Events & 
	    \begin{enumerate}[topsep=-1em,leftmargin=*]
		    \item[1.] The Administrator begins the cuPID process.
			\begin{enumerate}
				\item[2.] The Administrator is presented with a menu from the system. It has the options of: creating a new project, editing a project, launching the PPID, viewing PPID results.
				\item[3.] If the Administrator chooses to create a new project, the system brings up a menu for creating a new project (include use case \textbf{CreateNewProject}).
				\item[4.] If the Administrator chooses to edit a project, the system brings up a list of projects to edit (include use case \textbf{EditProject}).
				\item[5.] If the Administrator chooses to launch the PPID, the system brings up a list of projects (include use case \textbf{LaunchPPID}).
				\item[6.] If the Administrator chooses to view PPID results, the systems brings up a list of projects with PPID results (include use case \textbf{ViewPPIDResults}).
			\end{enumerate}
		\end{enumerate} \\

		Entry Conditions & \\

		Exit Conditions & \\

		Quality Requirements &
		\begin{itemize}[topsep=-1em,leftmargin=*]
		    \item The user must be operating cuPID as an Administrator. 
        \end{itemize} \\

		Traceability & F-05, F-06, F-07, F-08, F-09\\
        \toprule
    \end{tabu}
\end{center}

\begin{center}
	\begin{tabu} to \textwidth {>{\it}l X}
		\toprule
		Use Case Identifier & \createnewproject{} \\
		Name & {\bf CreateNewProject} \\
		Participating Actors & Initiated by Administrator \\
		Flow of Events & 
	    \begin{enumerate}[topsep=-1em,leftmargin=*]
		    \item[1.] The Administrator fills in the project details (include use case \textbf{EditProjectDetails}).
		    \begin{enumerate}
		        \item[2.] The system saves the new project.
		    \end{enumerate}
		\end{enumerate} \\

		Entry Conditions &
		\begin{itemize}[topsep=-1em,leftmargin=*]
		    \item The Administrator has chosen to create a new project.
        \end{itemize} \\

		Exit Conditions &
		\begin{itemize}[topsep=-1em,leftmargin=*]
		    \item A new Project has been created.
        \end{itemize} \\

		Quality Requirements & \\

		Traceability & F-05 \\
		\toprule
	\end{tabu}
\end{center}

\begin{center}
	\begin{tabu} to \textwidth {>{\it}l X}
		\toprule
		Use Case Identifier & \launchppid{} \\
		Name & {\bf LaunchPPID} \\
		Participating Actors & Initiated by Administrator \\
		Flow of Events & 
	    \begin{enumerate}[topsep=-1em,leftmargin=*]
		    \item[1.] The Administrator chooses a project on which to run the PPID.
		    \begin{enumerate}
		        \item[2.] The system runs the PPID on the chosen project.
		        \item[3.] The system displays the results of the PPID (include use case \textbf{ViewPPIDResults}).
		    \end{enumerate}
		\end{enumerate} \\

		Entry Conditions &
		\begin{itemize}[topsep=-1em,leftmargin=*]
		    \item The Administrator has chosen to run the PPID.
        \end{itemize} \\

		Exit Conditions &
		\begin{itemize}[topsep=-1em,leftmargin=*]
		    \item The PPID results are saved.
        \end{itemize} \\

		Quality Requirements &
		\begin{itemize}[topsep=-1em,leftmargin=*]
		    \item The PPID will take no longer than 5 seconds to complete.
        \end{itemize} \\

		Traceability & F-07, NF-07 \\
		\toprule
	\end{tabu}
\end{center}

\begin{center}
	\begin{tabu} to \textwidth {>{\it}l X}
		\toprule
		Use Case Identifier & \viewppidresults{} \\
		Name & {\bf ViewPPIDResults} \\
		Participating Actors & Initiated by Administrator \\
		Flow of Events & 
	    \begin{enumerate}[topsep=-1em,leftmargin=*]
		    \item[]
		    \begin{enumerate}
		        \item[1.] The system displays a menu with two options: view summary PPID results (include use case \textbf{ViewPPIDSummary}) or view detailed PPID results.
		        \item[2.] If the Administrator chooses to view the summary PPID results, the system displays the summary PPID results (include use case \textbf{ViewPPIDSummary}).
		        \item[3.] If the Administrator chooses to view the detailed PPID results, the system displays the detailed PPID results (include use case \textbf{ViewPPIDDetails}).
		    \end{enumerate}
		\end{enumerate} \\

		Entry Conditions &
		\begin{itemize}[topsep=-1em,leftmargin=*]
		    \item The Administrator has chosen to view the PPID results, or a new run of the PPID has just completed.
        \end{itemize} \\

		Exit Conditions & \\

		Quality Requirements & \\

		Traceability & F-08, F-09 \\
		\toprule
	\end{tabu}
\end{center}

\begin{center}
	\begin{tabu} to \textwidth {>{\it}l X}
		\toprule
		Use Case Identifier & \viewppidsummary{} \\
		Name & {\bf ViewPPIDSummary} \\
		Participating Actors & Initiated by Administrator \\
		Flow of Events & 
	    \begin{enumerate}[topsep=-1em,leftmargin=*]
		    \item[]
		    \begin{enumerate}
		        \item[1.] The system shows each group, including the students in each group.
		    \end{enumerate}
		\end{enumerate} \\

		Entry Conditions &
		\begin{itemize}[topsep=-1em,leftmargin=*]
		    \item The Administrator has chosen to view the summary PPID results.
        \end{itemize} \\

		Exit Conditions & \\

		Quality Requirements & \\

		Traceability & F-08 \\
		\toprule
	\end{tabu}
\end{center}

\begin{center}
	\begin{tabu} to \textwidth {>{\it}l X}
		\toprule
		Use Case Identifier & \viewppiddetails{} \\
		Name & {\bf ViewPPIDDetails} \\
		Participating Actors & Initiated by Administrator \\
		Flow of Events & 
	    \begin{enumerate}[topsep=-1em,leftmargin=*]
		    \item[]
		    \begin{enumerate}
		        \item[1.] The system displays the reasoning behind the groupings that were computed.
		    \end{enumerate}
		\end{enumerate} \\

		Entry Conditions &
		\begin{itemize}[topsep=-1em,leftmargin=*]
		    \item The Administrator has chosen to view the detailed PPID results.
        \end{itemize} \\

		Exit Conditions & \\

		Quality Requirements & \\

		Traceability & F-09 \\
		\toprule
	\end{tabu}
\end{center}

\begin{center}
	\begin{tabu} to \textwidth {>{\it}l X}
		\toprule
		Use Case Identifier & \editproject{} \\
		Name & {\bf EditProject} \\
		Participating Actors & Initiated by Administrator \\
		Flow of Events & 
	    \begin{enumerate}[topsep=-1em,leftmargin=*]
		    \item[1.] The Administrator selects a project to edit.
		    \begin{enumerate}
		        \item[2.] The system brings up a menu of details to edit (include use case \textbf{EditProjectDetails}).
		    \end{enumerate}
		    \item[3.] The Administrator chooses to finish editing the project.
		    \begin{enumerate}
		        \item[4.] The system saves the changed project details.
		    \end{enumerate}
		\end{enumerate} \\

		Entry Conditions &
		\begin{itemize}[topsep=-1em,leftmargin=*]
		    \item The Administrator has chosen to edit a project.
        \end{itemize} \\

		Exit Conditions &
		\begin{itemize}[topsep=-1em,leftmargin=*]
		    \item The changed project details have been saved.
        \end{itemize} \\

		Quality Requirements & \\

		Traceability & F-06 \\
		\toprule
	\end{tabu}
\end{center}

\begin{center}
	\begin{tabu} to \textwidth {>{\it}l X}
		\toprule
		Use Case Identifier & \editprojectdetails{} \\
		Name & {\bf EditProjectDetails} \\
		Participating Actors & Initiated by Administrator \\
		Flow of Events & 
	    \begin{enumerate}[topsep=-1em,leftmargin=*]
		    \item 
		\end{enumerate} \\

		Entry Conditions &
		\begin{itemize}[topsep=-1em,leftmargin=*]
		    \item 
        \end{itemize} \\

		Exit Conditions &
		\begin{itemize}[topsep=-1em,leftmargin=*]
		    \item 
        \end{itemize} \\

		Quality Requirements &
		\begin{itemize}[topsep=-1em,leftmargin=*]
		    \item 
        \end{itemize} \\

		Traceability & F-05, F-06, F-06-01, F-06-04 \\
		\toprule
	\end{tabu}
\end{center}

\begin{center}
	\begin{tabu} to \textwidth {>{\it}l X}
		\toprule
		Use Case Identifier & \editteamsize{} \\
		Name & {\bf EditTeamSize} \\
		Participating Actors & Initiated by Administrator \\
		Flow of Events & 
	    \begin{enumerate}[topsep=-1em,leftmargin=*]
		    \item 
		\end{enumerate} \\

		Entry Conditions &
		\begin{itemize}[topsep=-1em,leftmargin=*]
		    \item 
        \end{itemize} \\

		Exit Conditions &
		\begin{itemize}[topsep=-1em,leftmargin=*]
		    \item 
        \end{itemize} \\

		Quality Requirements &
		\begin{itemize}[topsep=-1em,leftmargin=*]
		    \item 
        \end{itemize} \\

		Traceability & F-06-01 \\
		\toprule
	\end{tabu}
\end{center}

\begin{center}
	\begin{tabu} to \textwidth {>{\it}l X}
		\toprule
		Use Case Identifier & \editprojectname{} \\
		Name & {\bf EditProjectName} \\
		Participating Actors & Initiated by Administrator \\
		Flow of Events & 
	    \begin{enumerate}[topsep=-1em,leftmargin=*]
		    \item 
		\end{enumerate} \\

		Entry Conditions &
		\begin{itemize}[topsep=-1em,leftmargin=*]
		    \item 
        \end{itemize} \\

		Exit Conditions &
		\begin{itemize}[topsep=-1em,leftmargin=*]
		    \item 
        \end{itemize} \\

		Quality Requirements &
		\begin{itemize}[topsep=-1em,leftmargin=*]
		    \item 
        \end{itemize} \\

		Traceability & F-06-04 \\
		\toprule
	\end{tabu}
\end{center}

\begin{center}
	\begin{tabu} to \textwidth {>{\it}l X}
		\toprule
		Use Case Identifier & \addstudent{} \\
		Name & {\bf AddStudent} \\
		Participating Actors & Initiated by Administrator \\
		Flow of Events & 
	    \begin{enumerate}[topsep=-1em,leftmargin=*]
		    \item 
		\end{enumerate} \\

		Entry Conditions &
		\begin{itemize}[topsep=-1em,leftmargin=*]
		    \item 
        \end{itemize} \\

		Exit Conditions &
		\begin{itemize}[topsep=-1em,leftmargin=*]
		    \item 
        \end{itemize} \\

		Quality Requirements &
		\begin{itemize}[topsep=-1em,leftmargin=*]
		    \item 
        \end{itemize} \\

		Traceability & F-06-02 \\
		\toprule
	\end{tabu}
\end{center}

\begin{center}
	\begin{tabu} to \textwidth {>{\it}l X}
		\toprule
		Use Case Identifier & \removestudent{} \\
		Name & {\bf RemoveStudent} \\
		Participating Actors & Initiated by Administrator \\
		Flow of Events & 
	    \begin{enumerate}[topsep=-1em,leftmargin=*]
		    \item 
		\end{enumerate} \\

		Entry Conditions &
		\begin{itemize}[topsep=-1em,leftmargin=*]
		    \item 
        \end{itemize} \\

		Exit Conditions &
		\begin{itemize}[topsep=-1em,leftmargin=*]
		    \item 
        \end{itemize} \\

		Quality Requirements &
		\begin{itemize}[topsep=-1em,leftmargin=*]
		    \item 
        \end{itemize} \\

		Traceability & F-06-03 \\
		\toprule
	\end{tabu}
\end{center}

\begin{center}
	\begin{tabu} to \textwidth {>{\it}l X}
		\toprule
		Use Case Identifier & \joinproject{} \\
		Name & {\bf JoinProject} \\
		Participating Actors & Initiated by Student \\
		Flow of Events & 
	    \begin{enumerate}[topsep=-1em, leftmargin=*]
		    \item[1.] The Student selects a project from the list of available projects.
		    \begin{enumerate}
		    		\item[2.] The system adds the player to the selected project.
		    \end{enumerate}
		\end{enumerate} \\

		Entry Conditions &
		\begin{itemize}[topsep=-1em, leftmargin=*]
		    \item The Student has selected to join a project.
        \end{itemize} \\

		Exit Conditions &
		\begin{itemize}[topsep=-1em, leftmargin=*]
		    \item The addition of the Student to a project has been stored permanently
        \end{itemize} \\

		Quality Requirements &
		\begin{itemize}[topsep=-1em, leftmargin=*]
		    \item The list of must only contains projects the student does not already belong to.
        \end{itemize} \\

		Traceability & F-01 \\
		\toprule
	\end{tabu}
\end{center}

\begin{center}
	\begin{tabu} to \textwidth {>{\it}l X}
		\toprule
		Use Case Identifier & \leaveproject{} \\
		Name & {\bf LeaveProject} \\
		Participating Actors & Initiated by Student \\
		Flow of Events & 
	    \begin{enumerate}[topsep=-1em, leftmargin=*]
		    \item[1.] The Student selects a project from a list of their joined projects.
		    \begin{enumerate}
		    		\item[2.] The system removes the Student from the selected project.
		    \end{enumerate}
		\end{enumerate} \\

		Entry Conditions &
		\begin{itemize}[topsep=-1em, leftmargin=*]
		    \item The Student has chosen to remove themselves from a project.
        \end{itemize} \\

		Exit Conditions &
		\begin{itemize}[topsep=-1em, leftmargin=*]
		    \item The removal of the Student from a project has been stored permanently.
        \end{itemize} \\

		Quality Requirements &
		\begin{itemize}[topsep=-1em, leftmargin=*]
		    \item Only the projects joined by the Student should be in the list.
        \end{itemize} \\

		Traceability & F-02 \\
		\toprule
	\end{tabu}
\end{center}

\begin{center}
	\begin{tabu} to \textwidth {>{\it}l X}
		\toprule
		Use Case Identifier & \editprofile{} \\
		Name & {\bf EditProfile} \\
		Participating Actors & Initiated by Student \\
		Flow of Events & 
	    \begin{enumerate}[topsep=-1em, leftmargin=*]
		    \item[]
		    \begin{enumerate}[topsep=-1em, leftmargin=*]
		    		\item[1.]
		    \end{enumerate}
		\end{enumerate} \\

		Entry Conditions &
		\begin{itemize}[topsep=-1em, leftmargin=*]
		    \item The Student has selected to edit their profile
        \end{itemize} \\

		Exit Conditions &
		\begin{itemize}[topsep=-1em, leftmargin=*]
		    \item 
        \end{itemize} \\

		Quality Requirements &
		\begin{itemize}[topsep=-1em, leftmargin=*]
		    \item 
        \end{itemize} \\

		Traceability &  \\
		\toprule
	\end{tabu}
\end{center}

\begin{center}
	\begin{tabu} to \textwidth {>{\it}l X}
		\toprule
		Use Case Identifier & \editpersonalvalues{} \\
		Name & {\bf EditPersonalValues} \\
		Participating Actors & Initiated by Student \\
		Flow of Events & 
	    \begin{enumerate}[topsep=-1em, leftmargin=*]
		    \item 
		\end{enumerate} \\

		Entry Conditions &
		\begin{itemize}[topsep=-1em, leftmargin=*]
		    \item 
        \end{itemize} \\

		Exit Conditions &
		\begin{itemize}[topsep=-1em, leftmargin=*]
		    \item 
        \end{itemize} \\

		Quality Requirements &
		\begin{itemize}[topsep=-1em, leftmargin=*]
		    \item 
        \end{itemize} \\

		Traceability &  \\
		\toprule
	\end{tabu}
\end{center}

\begin{center}
	\begin{tabu} to \textwidth {>{\it}l X}
		\toprule
		Use Case Identifier & \editdesiredvalues{} \\
		Name & {\bf EditDesiredValues} \\
		Participating Actors & Initiated by Student \\
		Flow of Events & 
	    \begin{enumerate}[topsep=-1em, leftmargin=*]
		    \item 
		\end{enumerate} \\

		Entry Conditions &
		\begin{itemize}[topsep=-1em, leftmargin=*]
		    \item 
        \end{itemize} \\

		Exit Conditions &
		\begin{itemize}[topsep=-1em, leftmargin=*]
		    \item 
        \end{itemize} \\

		Quality Requirements &
		\begin{itemize}[topsep=-1em, leftmargin=*]
		    \item 
        \end{itemize} \\

		Traceability &  \\
		\toprule
	\end{tabu}
\end{center}

\begin{center}
	\begin{tabu} to \textwidth {>{\it}l X}
		\toprule
		Use Case Identifier & \saveerror{} \\
		Name & {\bf SaveError} \\
		Participating Actors & Administrator \\
		Flow of Events & 
	    \begin{enumerate}[topsep=-1em,leftmargin=*]
		    \item The system notifies the Administrator that an error has occured with regards to the processing of
		    the current file.
		\end{enumerate} \\

		Entry Conditions &
		\begin{itemize}[topsep=-1em,leftmargin=*]
		    \item A file processing operation failed.
        \end{itemize} \\

		Exit Conditions &
		\begin{itemize}[topsep=-1em,leftmargin=*]
		    \item The operation is aborted.
        \end{itemize} \\

		Quality Requirements &
		\begin{itemize}[topsep=-1em,leftmargin=*]
		    \item 
        \end{itemize} \\

		Traceability & NF-02 \\
		\toprule
	\end{tabu}
\end{center}

\begin{center}
	\begin{tabu} to \textwidth {>{\it}l X}
		\toprule
		Use Case Identifier & \openerror{} \\
		Name & {\bf OpenError} \\
		Participating Actors & Administrator \\
		Flow of Events & 
	    \begin{enumerate}[topsep=-1em,leftmargin=*]
		    \item 
		\end{enumerate} \\

		Entry Conditions &
		\begin{itemize}[topsep=-1em,leftmargin=*]
		    \item 
        \end{itemize} \\

		Exit Conditions &
		\begin{itemize}[topsep=-1em,leftmargin=*]
		    \item 
        \end{itemize} \\

		Quality Requirements &
		\begin{itemize}[topsep=-1em,leftmargin=*]
		    \item 
        \end{itemize} \\

		Traceability & NF-02 \\
		\toprule
	\end{tabu}
\end{center}

\begin{center}
	\begin{tabu} to \textwidth {>{\it}l X}
		\toprule
		Use Case Identifier & \readerror{} \\
		Name & {\bf ReadError} \\
		Participating Actors & Administrator \\
		Flow of Events & 
	    \begin{enumerate}[topsep=-1em,leftmargin=*]
		    \item 
		\end{enumerate} \\

		Entry Conditions &
		\begin{itemize}[topsep=-1em,leftmargin=*]
		    \item 
        \end{itemize} \\

		Exit Conditions &
		\begin{itemize}[topsep=-1em,leftmargin=*]
		    \item 
        \end{itemize} \\

		Quality Requirements &
		\begin{itemize}[topsep=-1em,leftmargin=*]
		    \item 
        \end{itemize} \\

		Traceability & NF-02 \\
		\toprule
	\end{tabu}
\end{center}

\begin{center}
	\begin{tabu} to \textwidth {>{\it}l X}
		\toprule
		Use Case Identifier & \writeerror{} \\
		Name & {\bf WriteError} \\
		Participating Actors & Administrator \\
		Flow of Events & 
	    \begin{enumerate}[topsep=-1em,leftmargin=*]
		    \item 
		\end{enumerate} \\

		Entry Conditions &
		\begin{itemize}[topsep=-1em,leftmargin=*]
		    \item 
        \end{itemize} \\

		Exit Conditions &
		\begin{itemize}[topsep=-1em,leftmargin=*]
		    \item 
        \end{itemize} \\

		Quality Requirements &
		\begin{itemize}[topsep=-1em,leftmargin=*]
		    \item 
        \end{itemize} \\

		Traceability & NF-02 \\
		\toprule
	\end{tabu}
\end{center}

\begin{center}
	\begin{tabu} to \textwidth {>{\it}l X}
		\toprule
		Use Case Identifier & \closeerror{} \\
		Name & {\bf CloseError} \\
		Participating Actors & Administrator \\
		Flow of Events & 
	    \begin{enumerate}[topsep=-1em,leftmargin=*]
		    \item 
		\end{enumerate} \\

		Entry Conditions &
		\begin{itemize}[topsep=-1em,leftmargin=*]
		    \item 
        \end{itemize} \\

		Exit Conditions &
		\begin{itemize}[topsep=-1em,leftmargin=*]
		    \item 
        \end{itemize} \\

		Quality Requirements &
		\begin{itemize}[topsep=-1em,leftmargin=*]
		    \item 
        \end{itemize} \\

		Traceability & NF-02 \\
		\toprule
	\end{tabu}
\end{center}

\begin{center}
	\begin{tabu} to \textwidth {>{\it}l X}
		\toprule
		Use Case Identifier & \invalidinputerror{} \\
		Name & {\bf InvalidInputError} \\
		Participating Actors & Administrator, Student \\
		Flow of Events & 
	    \begin{enumerate}[topsep=-1em,leftmargin=*]
		    \item 
		\end{enumerate} \\

		Entry Conditions &
		\begin{itemize}[topsep=-1em,leftmargin=*]
		    \item 
        \end{itemize} \\

		Exit Conditions &
		\begin{itemize}[topsep=-1em,leftmargin=*]
		    \item 
        \end{itemize} \\

		Quality Requirements &
		\begin{itemize}[topsep=-1em,leftmargin=*]
		    \item 
        \end{itemize} \\

		Traceability & NF-02 \\
		\toprule
	\end{tabu}
\end{center}

\begin{center}
	\begin{tabu} to \textwidth {>{\it}l X}
		\toprule
		Use Case Identifier & \illegalminerror{} \\
		Name & {\bf IllegalMinError} \\
		Participating Actors & Administrator \\
		Flow of Events & 
	    \begin{enumerate}[topsep=-1em,leftmargin=*]
		    \item 
		\end{enumerate} \\

		Entry Conditions &
		\begin{itemize}[topsep=-1em,leftmargin=*]
		    \item 
        \end{itemize} \\

		Exit Conditions &
		\begin{itemize}[topsep=-1em,leftmargin=*]
		    \item 
        \end{itemize} \\

		Quality Requirements &
		\begin{itemize}[topsep=-1em,leftmargin=*]
		    \item 
        \end{itemize} \\

		Traceability & NF-02 \\
		\toprule
	\end{tabu}
\end{center}

\begin{center}
	\begin{tabu} to \textwidth {>{\it}l X}
		\toprule
		Use Case Identifier & \illegalmaxerror{} \\
		Name & {\bf IllegalMaxError} \\
		Participating Actors & Administrator \\
		Flow of Events & 
	    \begin{enumerate}[topsep=-1em,leftmargin=*]
		    \item 
		\end{enumerate} \\

		Entry Conditions &
		\begin{itemize}[topsep=-1em,leftmargin=*]
		    \item 
        \end{itemize} \\

		Exit Conditions &
		\begin{itemize}[topsep=-1em,leftmargin=*]
		    \item 
        \end{itemize} \\

		Quality Requirements &
		\begin{itemize}[topsep=-1em,leftmargin=*]
		    \item 
        \end{itemize} \\

		Traceability & NF-02 \\
		\toprule
	\end{tabu}
\end{center}

\begin{center}
	\begin{tabu} to \textwidth {>{\it}l X}
		\toprule
		Use Case Identifier & \projectexistserror{} \\
		Name & {\bf ProjectExistsError} \\
		Participating Actors & Administrator \\
		Flow of Events & 
	    \begin{enumerate}[topsep=-1em,leftmargin=*]
		    \item 
		\end{enumerate} \\

		Entry Conditions &
		\begin{itemize}[topsep=-1em,leftmargin=*]
		    \item 
        \end{itemize} \\

		Exit Conditions &
		\begin{itemize}[topsep=-1em,leftmargin=*]
		    \item 
        \end{itemize} \\

		Quality Requirements &
		\begin{itemize}[topsep=-1em,leftmargin=*]
		    \item 
        \end{itemize} \\

		Traceability & NF-02 \\
		\toprule
	\end{tabu}
\end{center}

\begin{center}
	\begin{tabu} to \textwidth {>{\it}l X}
		\toprule
		Use Case Identifier & \invalidstudenterror{} \\
		Name & {\bf InvalidStudentError} \\
		Participating Actors & Administrator \\
		Flow of Events & 
	    \begin{enumerate}[topsep=-1em,leftmargin=*]
		    \item 
		\end{enumerate} \\

		Entry Conditions &
		\begin{itemize}[topsep=-1em,leftmargin=*]
		    \item 
        \end{itemize} \\

		Exit Conditions &
		\begin{itemize}[topsep=-1em,leftmargin=*]
		    \item 
        \end{itemize} \\

		Quality Requirements &
		\begin{itemize}[topsep=-1em,leftmargin=*]
		    \item 
        \end{itemize} \\

		Traceability & NF-02 \\
		\toprule
	\end{tabu}
\end{center}

\begin{center}
	\begin{tabu} to \textwidth {>{\it}l X}
		\toprule
		Use Case Identifier & \insufficientstudentserror{} \\
		Name & {\bf InsufficientStudentsError} \\
		Participating Actors & Administrator \\
		Flow of Events & 
	    \begin{enumerate}[topsep=-1em,leftmargin=*]
		    \item 
		\end{enumerate} \\

		Entry Conditions &
		\begin{itemize}[topsep=-1em,leftmargin=*]
		    \item 
        \end{itemize} \\

		Exit Conditions &
		\begin{itemize}[topsep=-1em,leftmargin=*]
		    \item 
        \end{itemize} \\

		Quality Requirements &
		\begin{itemize}[topsep=-1em,leftmargin=*]
		    \item 
        \end{itemize} \\

		Traceability & NF-02 \\
		\toprule
	\end{tabu}
\end{center}

\begin{center}
	\begin{tabu} to \textwidth {>{\it}l X}
		\toprule
		Use Case Identifier & \invalidprojecterror{} \\
		Name & {\bf InvalidProjectError} \\
		Participating Actors & Student \\
		Flow of Events & 
	    \begin{enumerate}[topsep=-1em, leftmargin=*]
		    \item 
		\end{enumerate} \\

		Entry Conditions &
		\begin{itemize}[topsep=-1em, leftmargin=*]
		    \item 
        \end{itemize} \\

		Exit Conditions &
		\begin{itemize}[topsep=-1em, leftmargin=*]
		    \item 
        \end{itemize} \\

		Quality Requirements &
		\begin{itemize}[topsep=-1em, leftmargin=*]
		    \item 
        \end{itemize} \\

		Traceability & NF-02 \\
		\toprule
	\end{tabu}
\end{center}

\subsubsection{Object Model}


\subsubsection{Dynamic Model}


\end{document}
