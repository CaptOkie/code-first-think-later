\documentclass[12pt,letterpaper]{article}

\usepackage[T1]{fontenc}
\usepackage[margin=0.75in,headheight=1.5em]{geometry}
\usepackage{enumitem}
\usepackage{fancyhdr}
\usepackage{lastpage}
\usepackage{float}
\usepackage{tabu}
\usepackage{booktabs}
\usepackage{graphicx}
\usepackage{lmodern}
\usepackage[table]{xcolor}  
\usepackage[font=bf]{caption}

\begin{document}

\renewcommand\headrule{}

\pagestyle{fancy}
\fancyhf{}
\lfoot{COMP 3004}
\rfoot{\thepage/\pageref{LastPage}}
\cfoot{System Design Document}

% CUSTOM COMMANDS
\newcommand{\twodigits}[1]{\ifnum\value{#1}<10 0\fi\arabic{#1}}

\newcommand{\teamname}{Code First, Think Later}
\newcommand{\personone}{Kevin Hua}
\newcommand{\persontwo}{Hendrik Knoetze}
\newcommand{\personthree}{Juhandr\'e Knoetze}

%% FIGURE COMMANDS
\newcommand{\figurelabel}[1]{\label{figure:#1}}
\newcommand{\figureref}[1]{\textbf{Figure \ref{figure:#1}}}
%% END FIGURE COMMANDS

%% TABLE COMMANDS
\definecolor{thcolor}{RGB}{193,193,193}
\newcommand{\ccindent}{\hspace{1.5em}\hangindent=1.5em}
\newcommand{\tableheader}{\rowfont\bf\rowcolor{thcolor!30}}

\newcommand{\tablelabel}[1]{\label{table:#1}}
\newcommand{\tableref}[1]{\textbf{Table \ref{table:#1}}}
%% END TABLE COMMANDS

%% NODE COMMANDS
\newcounter{nodenum}
\renewcommand{\thenodenum}{\twodigits{nodenum}}
\newcommand{\nlabel}[1]{\refstepcounter{nodenum}\label{n:#1}}
\newcommand{\nref}[1]{\textbf{N-\ref{n:#1}}}
%% END NODE COMMANDS

%% SUBSYSTEM COMMANDS
\newcounter{subsystemnum}
\renewcommand{\thesubsystemnum}{\twodigits{subsystemnum}}
\newcommand{\sdlabel}[1]{\refstepcounter{subsystemnum}\label{sd:#1}}
\newcommand{\sdref}[1]{\textbf{SD-\ref{sd:#1}}}
%% END SUBSYSTEM COMMANDS

%% CLASS DIAGRAM COMMANDS
\newcounter{classdiagramnum}
\renewcommand{\theclassdiagramnum}{\twodigits{classdiagramnum}}
\newcommand{\cdlabel}[1]{\refstepcounter{classdiagramnum}\label{cd:#1}}
\newcommand{\cdref}[1]{\textbf{CD-\ref{cd:#1}}}
%% END CLASS DIAGRAM COMMANDS

%% DESIGN PATTERNS COMMANDS
\newcounter{designpatternnum}
\renewcommand{\thedesignpatternnum}{\twodigits{designpatternnum}}
\newcommand{\dplabel}[1]{\refstepcounter{designpatternnum}\label{dp:#1}}
\newcommand{\dpref}[1]{\textbf{DP-\ref{dp:#1}}}
%% END DESIGN PATTERNS COMMANDS

%% SERVICES COMMANDS
\newcounter{servicenum}
\renewcommand{\theservicenum}{\twodigits{servicenum}}
\newcommand{\sslabel}[1]{\refstepcounter{servicenum}\label{ss:#1}}
\newcommand{\ssref}[1]{\textbf{SS-\ref{ss:#1}}}
%% END SERVICES COMMANDS

%% OPERATIONS COMMANDS
\newcounter{operationnum}
\renewcommand{\theoperationnum}{\twodigits{operationnum}}
\newcommand{\oplabel}[1]{\refstepcounter{operationnum}\label{op:#1}}
\newcommand{\opref}[1]{\textbf{OP-\ref{op:#1}}}
%% END OPERATIONS COMMANDS

%% TABLE COMMANDS
\newcounter{dbtablenum}
\renewcommand{\thedbtablenum}{\twodigits{dbtablenum}}
\newcommand{\dtlabel}[1]{\refstepcounter{dbtablenum}\label{dt:#1}}
\newcommand{\dtref}[1]{\textbf{DT-\ref{dt:#1}}}
%% END TABLE COMMANDS

% TABLE STYLING
\everyrow{\hline}
\tabulinesep=0.5em

\setlist[itemize]{leftmargin=*,noitemsep,nolistsep}
\setlist[enumerate]{leftmargin=*,noitemsep,nolistsep}
% END TABLE STYLING

\thispagestyle{empty}

\begin{center}
	CARLETON UNIVERSITY
\end{center}

\vfill

\begin{center}
	{\fontsize{55pt}{55pt}\selectfont cuPID}
	\vspace{0.5em}\rule{\textwidth}{0.5pt}
	System Design Document
\end{center}

\vspace{5em}

\begin{center}
	\textbf{Team [\teamname{}]}\\
	\personone{}\\
	\persontwo{}\\
	\personthree{}
\end{center}

\vfill

\begin{center}
	Submitted to:\\
	Dr. Christine Laurendeau\\
	COMP 3004: Object Oriented Software Engineering\\
	School of Computer Science\\
	Carleton University
\end{center}

\vspace{2em}

\begin{center}
	\today
\end{center}

\newpage{}

\tableofcontents{}

\renewcommand{\listfigurename}{Figures}
\listoffigures

\renewcommand{\listtablename}{Tables}
\listoftables

\newpage{}

\section{Introduction}

\subsection{Purpose of System}

Team projects are intrinsically part of every student's academic life at some point or another. Many become resigned to losing the marks due to poor compatbility with assigned team members. This system has been designed to alleviate some of that pain by removing both the random factor and human error with regards to an educator's decision. Our system employs a special sorting algorithm that we have developed that combines psychological knowledge and research with computational reliability.

 Loosely speaking, the purpose of this system is to sort a group or class of people into groups of a specified size. However, more than just sorting people, our software takes into account a total of fifteen different characteristics of a person to determine the most optimal team that is based on more than just academic achievement. Here at [Code First, Think Later], we strongly believe that smart people don't have to like another smart person simply because they're smart.

\subsection{Overview of Document}

This report will provide an updated overview of various design decisions with regards to our system. This new system aims to correct all the shortcomings of the previous prototype, as well as smooth out several other issues that we encountered during the coding of the prototype. Futhermore, at this stage of development, it is time to revise and reorganize our prototype in such a way that it adheres more closely with both a reliable architectural style and appropriate design styles. 

Beginning with a full decomposition of our prototype, we will then show a modified decomposition wherein poor design choices have been fixed. Following these decompositions, our report will feature our chosen strategies with regards to architectural and design styles. We will also be covering the details of our persistent storage system in this section. Afterwards, we will revisit the subsystems that we have decomposed in the first section and provide a detailed set of descriptions as to what services each offers in the grand scheme of things. Finally, we will end the report with a UML style set of class diagrams for each class.

\vspace{1em}

\noindent Best regards,

\vspace{1em}

\begin{center}
	\includegraphics[scale=0.4]{imgs/logo.png} \\ \footnotesize{(Team [Code First, Think Later])}
\end{center}

\section{Subsystem Decomposition}

This section will be three-pronged - firstly, we will decompose the current working prototype into its corresponding set of logical subsystem. Following, we will introduce a more complete system decomposition, wherein we will include aspects that are not yet implemented in the current cuPID prototype. Also in this part, we will possibly modify existing subsystmes from the prototype to promote a higher degree of cohesion and minimal coupling. Finally, we will discuss the changes that we have decided on and how they impact the software as a whole. 

\subsection{Phase 1 Prototype Decomposition}
\begin{figure}[H]
	\centering{}
	\includegraphics[scale=0.55]{imgs/d3/class-diagram-prototype.png}
	\caption{Subsystem Decomposition (Prototype)}
	\figurelabel{proto-decomp}
\end{figure}
\subsection{System Decomposition}
\begin{figure}[H]
	\centering{}
	\includegraphics[scale=0.6]{imgs/d3/class-diagram-decomp.png}
	\caption{Subsystem Decomposition}
	\figurelabel{subsys-decomp}
\end{figure}
\subsection{Design Evolution}
\subsubsection{Prototype Design Decisions}
\subsubsection{Updated Design Decisions}
\section{Design Strategies}

Similar to the previous section, this section is also divided into three major parts: architectural style, persistent data management, and design patterns. In the first part, we will describe our chosen architectural style and explain the benefits to using this particular style with regards to our system. We will also include UML deployment diagrams as a visual representation of the architecture of our software. Following this part, we will discuss our design choices with regards to our persistent data management. We will also be covering some particular cases here, such as duplicate records. Finally, we will discuss the our chosen design patterns and why they are appropriate given our decisions form the other parts in this section.

\subsection{Hardware/Software Mapping}
\subsubsection{Overview}

Our system revolves heavily upon a central storage, or repository. Other than the repository, we make use of several subsystems. Simply with this in mind, we can instantly rule out the {\it pipe and filter} architectural style - while we do indeed have subsystems (filters) that associate with each other (pipes), our subsystems are fairly dependent on one another. Furthermore, our subsystems are responsible for more than simply processing a set of inputs to get a set of outputs. 

We can also rule out the four-tier architectural style right away - our system is entirely local, thus a separate client and server layer are unnecessary. We use a login-type system, and the UI is common to any user that logs in. The only difference would be that the displayed information changes, as well as some usability. That said, this type of functionality would best fit as multiple partitions within a layer. As a result, the difference in the student and admin users should be as sister partitions within the same layer.

\subsubsection{Three-Tier Architecture}
\subsubsection{Model-View Controller Architecture}
\subsubsection{Repository Architecture}

\subsection{Persistent Data Management}
\subsubsection{Overview}

To store the data for our system, a relational database is being used. Our relational database is built in the SQLite architecture. Multiple tables are used to store related information together, such as the \textit{Students} table.

\begin{figure}[H]
	\centering{}
	\includegraphics[scale=0.45]{imgs/d3/db/database-schema-diagram.png}
	\caption{Database Schema}
	\figurelabel{db-schema}
\end{figure}

\begin{table}[H]
	\caption{Tables in Database} \tablelabel{db-table}
	\begin{tabu} to \textwidth {l >{\it}l X}
		\tableheader{}ID & Name & Description \\
		\dtlabel{students}\dtref{students}     & students   & Stores the Student entity object. \\
		\dtlabel{admins}\dtref{admins}         & admins     & Stores the Admin entity object. \\
		\dtlabel{projects}\dtref{projects}     & projects   & Stores the Project entity object. \\
		\dtlabel{enrollment}\dtref{enrollment} & enrollment & Stores the association between the Student and Project entity objects,
		                                                      which is refers to a student's enrollment into a project. \\
		\dtlabel{questions}\dtref{questions}   & questions  & Stores the Question entity object. \\
		\dtlabel{answers}\dtref{answers}       & answers    & Stores the Answer entity object. \\
		\dtlabel{responses}\dtref{responses}   & responses  & Stores the association between the Student and Question entity objects. 
		                                                      which refers to what answer a student has selected for a question. \\
	\end{tabu}
\end{table}

\begin{table}[H]
	\caption{Students Table (\dtref{students})} \tablelabel{students-table}
	\begin{tabu} to \textwidth {l l l X}
		\tableheader{}Column & Type & Constraint & Description \\
		id   & Integer & Primary Key & The unique ID given to the student. \\
		name & Text    &             & The name of the student. \\
	\end{tabu}
\end{table}

\begin{table}[H]
	\caption{Admins Table (\dtref{admins})} \tablelabel{admins-table}
	\begin{tabu} to \textwidth {l l l X}
		\tableheader{}Column & Type & Constraint & Description \\
		id   & Integer & Primary Key & The unique ID given to the administrator. \\
		name & Text    &             & The name of the administrator. \\
	\end{tabu}
\end{table}

\begin{table}[H]
	\caption{Projects Table (\dtref{projects})} \tablelabel{projects-table}
	\begin{tabu} to \textwidth {l l l X}
		\tableheader{}Column & Type & Constraint & Description \\
		name           & Text    & Primary Key & The unique name given to a project. \\
		max\_grp\_size & Integer &             & The maximum group size of a project. \\
		min\_grp\_size & Integer &             & The minimum group size of a project. \\
	\end{tabu}
\end{table}

\begin{table}[H]
	\caption{Enrollment Table (\dtref{enrollment})} \tablelabel{enrollment-table}
	\begin{tabu} to \textwidth {l l X X[3]}
		\tableheader{}Column & Type & Constraint & Description \\
		stu     & Integer & Primary Key,\newline
		                    Foreign Key          & References \textit{id} of the Students table. \\
		project & Text    & Primary Key,\newline
		                    Case Insensitive     &  \\
	\end{tabu}
\end{table}

\begin{table}[H]
	\caption{Questions Table (\dtref{questions})} \tablelabel{questions-table}
	\begin{tabu} to \textwidth {l >{\it}l X}
		\tableheader{}Column & Constraint & Description \\
	\end{tabu}
\end{table}

\begin{table}[H]
	\caption{Answers Table (\dtref{answers})} \tablelabel{answers-table}
	\begin{tabu} to \textwidth {l >{\it}l X}
		\tableheader{}Column & Constraint & Description \\
	\end{tabu}
\end{table}

\begin{table}[H]
	\caption{Responses Table (\dtref{responses})} \tablelabel{responses-table}
	\begin{tabu} to \textwidth {l >{\it}l X}
		\tableheader{}Column & Constraint & Description \\
	\end{tabu}
\end{table}

\subsubsection{Reasoning}
\subsection{Design Patterns}
\subsubsection{Overview}
While organizing code with an eye to the larger picture - architectural styles - is vitally important, the smaller scale design patterns fall not far behind. Design patterns are largely independent of the architectural style insofar as nearly all design patterns can be implemented in any of the architectural patterns. They usually are used to minimize coupling across layers or to maximize cohesion across sister subsystems (partitions). This section will go over several different design patterns and explain either why we decided to use them or why we forgoed their implementation. There are three main categories of design patterns that we will include in this breakdown: creational patterns, structural patterns, and behavioural patterns.
\subsubsection{Creational Patterns}
Patterns in this section affect how objects get created. 
\subsubsection*{Abstract Factory}
We have decided to implement this design pattern in our system. Add more here later.
\subsubsection{Structural Patterns}
\subsubsection*{Adapter}
This design pattern is ill suited to our particular system since we will be developing the entirety of our system. Since we will not need to write code that adapts between newer and older code, this pattern would be counter-intuitive.
\subsubsection*{Bridge}
\subsubsection*{Facade}
\subsubsection*{Proxy}
We have chosen to implement this design pattern in our system. By empassing several of our entity objects in a proxy design, we can implement a `lazy' load-type system in that the system will only load the information that we directly ask for - nothing extra. This has the enormous benefit of reducing the memory footprint. The time tradeoff is comparably minimal, even neglible. 
\subsubsection{Behavioural Patterns}
\subsubsection*{Command}
\subsubsection*{Observer}
\subsubsection*{Strategy}
\section{Subsystem Services}
\subsection{Overview}
\subsection{Services}

In this section, we give a list and description of each of the services that our subsystems offer. They will be divided into three separate tables - one for each of our subsystems. The below table, \tableref{service-sub-repo}, covers the services offered by our Repository Subsystem (\sdref{repo}).

\begin{table}[H]
\caption{Services Offered by Repository Subsystem (\sdref{repo})} \tablelabel{service-sub-repo}
\begin{tabu} to \textwidth {l >{\it}l X}
	\tableheader{}ID & Service & Description\\
	\sslabel{service1}\ssref{service1} & service1 & desc\\
	\sslabel{service2}\ssref{service2} & service2 & desc\\
\end{tabu}
\end{table}

The following table encompasses all the services offered by the Student Subsystem (\sdref{stu}). 

\begin{table}[H]
\caption{Services Offered by Student Subsystem (\sdref{stu})} \tablelabel{service-sub-stu}
\begin{tabu} to \textwidth {l >{\it}l X}
	\tableheader{}ID & Service & Description\\
	\sslabel{service3}\ssref{service3} & service3 & desc\\
	\sslabel{service4}\ssref{service4} & service4 & desc\\
\end{tabu}
\end{table}

The following table showcases all the different services offered by the Administrator Subsytem (\sdref{admin}).

\begin{table}[H]
\caption{Services Offered by Administrator Subsystem (\sdref{admin})} \tablelabel{service-sub-admin}
\begin{tabu} to \textwidth {l >{\it}l X}
	\tableheader{}ID & Service & Description\\
	\sslabel{service5}\ssref{service5} & service5 & desc\\
	\sslabel{service6}\ssref{service6} & service6 & desc\\
\end{tabu}
\end{table}

\subsection{Operations}

In this section, we focus on the Operations that each of the services described in the above section offer. For added traceability, we have also included the specific class that provides each of the operations, as well as a description of each.

\begin{table}[H]
\caption{Operations Offered in Service1 (\ssref{service1})} \tablelabel{operation-service1}
\begin{tabu} to \textwidth {l >{\it}l l X}
	\tableheader{}ID & Operation & Class & Description\\
	\oplabel{operation1}\opref{operation1} & operation1 & classA & desc\\
	\oplabel{operation2}\opref{operation2} & operation2 & classB & desc\\
\end{tabu}
\end{table}

\begin{table}[H]
\caption{Operations Offered in Service2 (\ssref{service2})} \tablelabel{operation-service2}
\begin{tabu} to \textwidth {l >{\it}l l X}
	\tableheader{}ID & Operation & Class & Description\\
	\oplabel{operation3}\opref{operation3} & operation1 & classA & desc\\
	\oplabel{operation4}\opref{operation4} & operation2 & classB & desc\\
\end{tabu}
\end{table}

\begin{table}[H]
\caption{Operations Offered in Service3 (\ssref{service3})} \tablelabel{operation-service3}
\begin{tabu} to \textwidth {l >{\it}l l X}
	\tableheader{}ID & Operation & Class & Description\\
	\oplabel{operation5}\opref{operation5} & operation1 & classA & desc\\
	\oplabel{operation6}\opref{operation6} & operation2 & classB & desc\\
\end{tabu}
\end{table}

\begin{table}[H]
\caption{Operations Offered in Service4 (\ssref{service4})} \tablelabel{operation-service4}
\begin{tabu} to \textwidth {l >{\it}l l X}
	\tableheader{}ID & Operation & Class & Description\\
	\oplabel{operation7}\opref{operation7} & operation1 & classA & desc\\
	\oplabel{operation8}\opref{operation8} & operation2 & classB & desc\\
\end{tabu}
\end{table}

\begin{table}[H]
\caption{Operations Offered in Service5 (\ssref{service5})} \tablelabel{operation-service5}
\begin{tabu} to \textwidth {l >{\it}l l X}
	\tableheader{}ID & Operation & Class & Description\\
	\oplabel{operation9}\opref{operation9} & operation1 & classA & desc\\
	\oplabel{operation10}\opref{operation10} & operation2 & classB & desc\\
\end{tabu}
\end{table}

\section{Class Interfaces}

\begin{table}[H]
\caption{} 
\begin{tabu} to \textwidth {l >{\it}l X}
	\tableheader{}ID & Class Name & Description \\
	\cdlabel{user}\cdref{user} & User & The user class contains only a single attribute - the unique ID. It provides the archetype for a generic user - the only defining feature being the unique identifier.\\
	\cdlabel{student}\cdref{student} & Student & The Student class contains several attributes: a unique ID, inherited from the User class, a collection of responses, and their join-status with regards to projects. Their methods revolve around the handling of their profiles, such as modifying response values. This class inherits from the User class. \\
	\cdlabel{realstudent}\cdref{realstudent} & RealStudent & The RealStudent class inherits from the Student class and contains the actual information pulled from the StudentStorage object. It is relatively hidden from the rest of the system, though the data can be accessed through the ProxyStudent object.\\
	\cdlabel{proxystudent}\cdref{proxystudent} & ProxyStudent & The ProxyStudent class also inherits from the Student class and provides a proxy or interface to the RealStudent object for the rest of the system. The idea here is to promote `lazy' loading with regards to the information stored in the RealStudent object.\\
	\cdlabel{studentstorage}\cdref{studentstorage} & StudentStorage & The StudentStorage class is a class that provides a set of methods to pull/push data related to the Student user from/to the relational database. The RealStudent object is the only object that directly interacts with this object.\\
	\cdlabel{admin}\cdref{admin} & Admin & The Admin class contains only a single attribute - the unique ID, inherited from the User class - and several methods revolving around the handling of Project objects. This class inherits from the User class.  \\
	\cdlabel{realadmin}\cdref{realadmin} & RealAdmin & The RealAdmin class inherits from the Admin class and contains the actual information pulled from the AdminStorage object. It is relatively hidden from the rest of the system, though the data contained can only be accessed through the ProxyAdmin object. \\
	\cdlabel{proxyadmin}\cdref{proxyadmin} & ProxyAdmin & The ProxyAdmin class also inherits from the Admin class and provides a proxy or interface to the RealAdmin object for the rest of the system. The idea here is to promote `lazy' loading with regards to the information stored in the RealAdmin object.\\
	\cdlabel{adminstorage}\cdref{adminstorage} & AdminStorage & The AdminStorage class is a class that provides a set of methods to pull/push data related to the Admin user from/to the relational database. The RealAdmin object is the only object that directly interacts with this object.\\
\end{tabu}
\end{table}

\begin{center}
\begin{tabu} to \textwidth {l >{\it}l X}
	\tableheader{}ID & Class Name & Description \\
	\cdlabel{project}\cdref{project} & Project & The Project class contains two attributes: a GroupSize object as an attribute and a collection of Student objects. Its methods revolve around simply returning the collection of Students.\\
	\cdlabel{realproject}\cdref{realproject} & RealProject & The RealProject class inherits from the Project class and contains the actual information pulled from the ProjectStorage object. It is relatively hidden from the rest of the system, though the data can be accessed through the ProxyProject object.\\
	\cdlabel{proxyproject}\cdref{proxyproject} & ProxyProject & The ProxyProject also inherits from the Project class and provides a proxy or interface to the RealProject object for the rest of the system. The idea here is to promote `lazy' loading with regards to the information stored in the RealProject object.\\
	\cdlabel{projectstorage}\cdref{projectstorage} & ProjectStorage & The ProjectStorage class is a class that provides a set of methods to pull/push data related to Projects from/to the relational database. The RealProject object is the only object that directly interacts with this object.\\
	\cdlabel{question}\cdref{question} & Question & The Question class contains an attribute that contains the a String representation of the questions used in the algorithm. Also as an attribute, the Question class contains an integer attribute that stores the scale of possible responses. \\
	\cdlabel{realquestion}\cdref{realquestion} & RealQuestion & The RealQuestion class inherits from the Question class and contains the actual information pulled from the QuestionStorage object. It is relatively hidden from the rest of the system, though the data can be accessed through the ProxyQuestion object.\\
	\cdlabel{proxyquestion}\cdref{proxyquestion} & ProxyQuestion & The ProxyQuestion also inherits from the Question class and provides a proxy or interface to the RealQuestion object for the rest of the system. The idea here is to promote `lazy' loading with regards to the information stored in the RealQuestion object.\\
	\cdlabel{questionstorage}\cdref{questionstorage} & QuestionStorage & The QuestionStorage class is a class that provides a set of methods to pull/push data related to Questions from/to the relational database. The RealQuestion object is the only object that directly interacts with this object.\\
	\cdlabel{answer}\cdref{answer} & Answer & The Answer class contains three attributes: the question number, a unique identifier, and the set of answers. The methods this class provides revolve around simply returning the required set of answers.\\
\end{tabu}
\end{center}

\begin{center}
\begin{tabu} to \textwidth {l >{\it}l X}
	\tableheader{}ID & Class Name & Description \\
	\cdlabel{group}\cdref{group} & Group & \\
	\cdlabel{groupsize}\cdref{groupsize} & GroupSize & The GroupSoze class contains two attributes - the minimum group size and the maximum group size. This class offers no methods other than standard getters and setters for each attribute. \\
	\cdlabel{grouper}\cdref{grouper} & Grouper & \\
	\cdlabel{matcher}\cdref{matcher} & Matcher & \\
	\cdlabel{splitsmallest}\cdref{splitsmallest} & SplitSmallest & \\
	\cdlabel{percentdistance}\cdref{percentdistance} & PercentDistance & \\
	\cdlabel{ppidcontrol}\cdref{ppidcontrol} & PPIDControl & desc \\
	\cdlabel{logincontrol}\cdref{logincontrol} & LoginControl & desc \\
	\cdlabel{projectcontrol}\cdref{projectcontrol} & ProjectControl\\
\end{tabu}
\end{center}

\begin{center}
\begin{tabu} to \textwidth {l >{\it}l X}
	\tableheader{}ID & Class Name & Description \\
	\cdlabel{usercontrol}\cdref{usercontrol} & UserControl & \\
	\cdlabel{studenthomecontrol}\cdref{studenthomecontrol} & StudentHomeControl & desc \\
	\cdlabel{adminhomecontrol}\cdref{adminhomecontrol} & AdminHomeControl & desc \\
	\cdlabel{profilecontrol}\cdref{profilecontrol} & ProfileControl & desc \\
	\cdlabel{questioncontrol}\cdref{questioncontrol} & QuestionControl & desc \\
	\cdlabel{questionscontrol}\cdref{questionscontrol} & QuestionsControl & desc \\
	\cdlabel{userconfact}\cdref{userconfact} & UserControlFactory & \\
	\cdlabel{userstorage}\cdref{userstorage} & UserStorage & \\
	\cdlabel{loginform}\cdref{loginform} & LoginForm & \\

\end{tabu}
\end{center}

\begin{center}
\begin{tabu} to \textwidth {l >{\it}l X}
	\tableheader{}ID & Class Name & Description \\
	\cdlabel{questionsform}\cdref{questionsform} & QuestionsForm & \\
	\cdlabel{questionoption}\cdref{questionoption} & QuestionOption & \\
	\cdlabel{profileform}\cdref{profileform} & ProfileForm & \\
	\cdlabel{studentform}\cdref{studentform} & StudentForm & \\
	\cdlabel{adminform}\cdref{adminform} & AdminForm & \\
	\cdlabel{projectform}\cdref{projectform} & ProjectForm & \\
	\cdlabel{ppidform}\cdref{ppidform} & PPIDForm & \\
\end{tabu}
\end{center}

\end{document}
