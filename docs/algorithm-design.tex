\documentclass[12pt,letterpaper]{article}

\usepackage[T1]{fontenc}
\usepackage[margin=0.75in,headheight=1.5em]{geometry}
\usepackage{enumitem}
\usepackage{fancyhdr}
\usepackage{lastpage}
\usepackage{float}
\usepackage{tabu}
\usepackage{booktabs}
\usepackage{graphicx}
\usepackage{lmodern}
\usepackage[table]{xcolor}  
\usepackage[font=bf]{caption}

\begin{document}

\renewcommand\headrule{}

\pagestyle{fancy}
\fancyhf{}
\lfoot{COMP 3004}
\rfoot{\thepage/\pageref{LastPage}}
\cfoot{Algorithm Design Document}

% CUSTOM COMMANDS
\newcommand{\twodigits}[1]{\ifnum\value{#1}<10 0\fi\arabic{#1}}

\newcommand{\teamname}{Code First, Think Later}
\newcommand{\personone}{Kevin Hua}
\newcommand{\persontwo}{Hendrik Knoetze}
\newcommand{\personthree}{Juhandr\'e Knoetze}

%% FIGURE COMMANDS
\newcommand{\figurelabel}[1]{\label{figure:#1}}
\newcommand{\figureref}[1]{\textbf{Figure \ref{figure:#1}}}
%% END FIGURE COMMANDS

%% TABLE COMMANDS
\definecolor{thcolor}{RGB}{193,193,193}
\newcommand{\ccindent}{\hspace{1.5em}\hangindent=1.5em}
\newcommand{\tableheader}{\rowfont\bf\rowcolor{thcolor!30}}

\newcommand{\tablelabel}[1]{\label{table:#1}}
\newcommand{\tableref}[1]{\textbf{Table \ref{table:#1}}}
%% END TABLE COMMANDS

% TABLE STYLING
\everyrow{\hline}
\tabulinesep=0.5em

\setlist[itemize]{leftmargin=*,noitemsep,nolistsep}
\setlist[enumerate]{leftmargin=*,noitemsep,nolistsep}
% END TABLE STYLING

\thispagestyle{empty}

\begin{center}
	CARLETON UNIVERSITY
\end{center}

\vfill

\begin{center}
	{\fontsize{55pt}{55pt}\selectfont cuPID}
	\vspace{0.5em}\rule{\textwidth}{0.5pt}
	Algorithm Design Document
\end{center}

\vspace{5em}

\begin{center}
	\textbf{Team [\teamname{}]}\\
	\personone{}\\
	\persontwo{}\\
	\personthree{}
\end{center}

\vfill

\begin{center}
	Submitted to:\\
	Dr. Christine Laurendeau\\
	COMP 3004: Object Oriented Software Engineering\\
	School of Computer Science\\
	Carleton University
\end{center}

\vspace{2em}

\begin{center}
	\today
\end{center}

\newpage{}

\tableofcontents{}

\renewcommand{\listfigurename}{Figures}
\listoffigures

\renewcommand{\listtablename}{Tables}
\listoftables

\newpage{}

\section{Introduction}

Large-scale projects have always required a team to develop, design, and maintain. Unfortunately, it is rarely the case that team members are chosen based on their personalities or other such personal qualities - when you get right down to what's important to the employer, skills are paramount. Sure, the quirky, good-natured person might be favored over the quieter, unsocial person, but if that quieter person eclipsed the former with regards to talent, then they get chosen. Regrettably, the end result becomes a team of peerless talent that clash heavily on all other aspects. Maybe they're all `alpha' personalities and cannot handle not being either the center of attention or in charge. Maybe they despise the idea of sharing code with each other. Either way, the project suffers in quality because of this. Our firm, [Code First, Think Later], would like to propose in this document a potential algorithm that could solve all these problems.

\subsection{Purpose of Algorithm}

Our PPID (Project-Partner-IDentification) algorithm's purpose, as the name suggests, is, given a pool of potential team members, to sort those members into the optimal teams based on several factors including relevant skills, personality traits, leadership qualities, and degree of availability. The basis upon which the algorithm sorts the members is a set of qualifications given by each member. The qualifications themselves can be divided into two sections: the qualifications that they have, and the qualifications that they desire in a group member. By incorporating both these sets, we can more confidently ensure that the groups fit perfectly.

\subsection{Overview of Document}

This document will start with basic, simple details and slowly build upwards in complexity. As such, we begin with the Qualifications section where explanations and descriptions of each of our proposed qualifications are outlined. Furthermore, we have included the scale of possible responses for each of the qualifications and assigned each a unique ID to allow maximum traceability throughout this document. Following the Qualifications section, you will find the Rules section where the algorithmic logic gets explained. Once again, each rule is assigned a unique ID. Finally, the last section will pull all the parts together and explain how the rules, when applied to the qualifications, will give optimal and consistent results.

\vspace{1em}

\noindent Best regards,

\vspace{1em}

\begin{center}
	\includegraphics[scale=0.4]{imgs/logo.png} \\ \footnotesize{(Team [Code First, Think Later])}
\end{center}

\vspace{1em}

\section{Proposed Algorithm}

\subsection{Overview}

In this section, {\bf Section 2}, we will be outlining all the aspects of our algorithm. The section will be divided into two parts: the qualifications and the rules. In the former part, we will go into heavy detail with regards to the qualifications that we used, whereas the latter part will focus on the different rules we're employing to adjust and transform the results of the qualifications into more palpitable results (represented as percentages). 

\subsection{Qualifications}

\subsubsection{Overview of Qualifications}

The following table, \tableref{oq}, gives a simple overview of the different qualifications that we are accounting for in our algorithms. Alongside each, there is a unique ID, as well as a category for that particular qualification. We have included four different categories: skills, leadership, personality, and availability. Finally, you will find a short description of each qualification.

\begin{table}[H]
	\caption{Overview of Qualifications}\tablelabel{oq}
	\begin{tabu} to \textwidth {>{\bf}l >{\it}l X}
	    \tableheader{}Identifier & Category/Class & Description\\
		QF-01 & Skills & Grade in COMP 2401.\\
		QF-02 & Skills & Grade in COMP 2404.\\
		QF-03 & Skills & Report writing aptitude.\\
		QF-04 & Skills & Debugging aptitude.\\
		QF-05 & Leadership & Comfort in taking charge. \\
		QF-06 & Leadership & Importance of organization. \\
		QF-07 & Leadership & Decision-making ability. \\
		QF-08 & Leadership & Tendency towards initiatiation. \\
		QF-09 & Personality & Procrastination tendency. \\
		QF-10 & Personality & Sociability level. \\
		QF-11 & Personality & Preferred working style. \\
		QF-12 & Personality & Punctuality level. \\
		QF-13 & Availability & Daily availability. \\
		QF-14 & Availability & Weekly availability. \\
		QF-15 & Availability & Preferred mode of communication. \\
	\end{tabu}
\end{table}

\subsubsection{Detailed Breakdown of Qualifications}

\tableref{dq01} below gives a more detailed explanation of {\bf QF-01}. It explains the importance of the COMP 2401 grade in the scheme of generating teams for Software Engineering.

\begin{table}[H]
	\caption{Detailed Breakdown of QF-01}\tablelabel{dq01}
	\begin{tabu} to \textwidth {>{\it}l X}
		\toprule
		Qualification Identifier & {\bf QF-01}\\
		Category & Skills \\
		Phrasing of Qualification & What grade did you receive in COMP 2401? \\
		Scale for Responses &
		\begin{minipage}[t]{\linewidth}
			\begin{enumerate}
				\item[1.] D- to D+
				\item[2.] C- to C+
				\item[3.] B- to B+
				\item[4.] A- to A+
			\end{enumerate}
		\end{minipage}\\
		Reasoning & COMP 2401, Introduction to Systems Programming, is a second year course whose material covers some of the basic but essential components of systems programming. These components include, but are not limited to: memory management, program design, and data types. Without the skills learned in this class, software may be poorly managed with regards to memory allocations, file divisions, and use of the dreaded global variables.\\
		References & Laurendeau\cite{claurend1}\\
		\toprule
	\end{tabu}
\end{table}

\tableref{dq02} below gives a more detailed explanation of {\bf QF-02}. It explains the importance of the COMP 2404 grade in the scheme of generating teams for Software Engineering.

\begin{table}[H]
	\caption{Detailed Breakdown of QF-02}\tablelabel{dq02}
	\begin{tabu} to \textwidth {>{\it}l X}
		\toprule
		Qualification Identifier & {\bf QF-02}\\
		Category & Skills \\
		Phrasing of Qualification & What grade did you receive in COMP 2404? \\
		Scale for Responses &
		\begin{minipage}[t]{\linewidth}
			\begin{enumerate}
				\item[1.] D- to D+
				\item[2.] C- to C+
				\item[3.] B- to B+
				\item[4.] A- to A+
			\end{enumerate}
		\end{minipage}\\
		Reasoning & COMP 2404, Introduction to Software Engineering, is a crucial second year course that covers nearly all the important parts pertaining to the software engineering life-cycle. Components include: the concepts of data abstraction, UML notation, design patterns, and fault/exception handling.\\
		References & Laurendeau\cite{claurend2}\\
		\toprule
	\end{tabu}
\end{table}

\newpage{}

\tableref{dq03} below gives a more detailed explanation of {\bf QF-03}. It explains the importance of report writing in the software development lifecycle.

\begin{table}[H]
	\caption{Detailed Breakdown of QF-03}\tablelabel{dq03}
	\begin{tabu} to \textwidth {>{\it}l X}
		\toprule
		Qualification Identifier & {\bf QF-03}\\
		Category & Skills \\
		Phrasing of Qualification & How confident and/or comfortable are you in report writing? \\
		Scale for Responses &
		\begin{minipage}[t]{\linewidth}
			\begin{enumerate}
				\item[1.] Very uncomfortable/zero confidence. 
				\item[2.] Uncomfortable/low confidence.
				\item[3.] It's okay I guess.
				\item[4.] I enjoy writing!
				\item[5.] I'm very comfortable/confident.
			\end{enumerate}
		\end{minipage}\\
		Reasoning & Report writing is typically seen as a crucial University skill to have. For the purpose of Software Engineering, it becomes important when it comes time to start writing up the deliverables. Knowing how to punctuate, as well as how to structure sentences, are vitally important.\\
		References & \\
		\toprule
	\end{tabu}
\end{table}

\tableref{dq04} below gives a more detailed explanation of {\bf QF-04}. It explains the importance of debugging software in the software development lifecycle.

\begin{table}[H]
	\caption{Detailed Breakdown of QF-04}\tablelabel{dq04}
	\begin{tabu} to \textwidth {>{\it}l X}
		\toprule
		Qualification Identifier & {\bf QF-04}\\
		Category & Skills \\
		Phrasing of Qualification & How proficient are you at software debugging? \\
		Scale for Responses &
		\begin{minipage}[t]{\linewidth}
			\begin{enumerate}
				\item[1.] I cannot debug.
				\item[2.] I have a lot of trouble with debugging.
				\item[3.] I can usually figure things out eventually.
				\item[4.] I have a fair idea of where the bugs usually happen so I'm pretty fast at debugging!
				\item[5.] I'm very good at debugging software.
			\end{enumerate}
		\end{minipage}\\
		Reasoning & It's nearly impossible that a given software runs flawlessly the first time it gets run. Debugging software is something of an art! Being able to break a program to assure that all test cases work is vital.\\
		References & \\
		\toprule
	\end{tabu}
\end{table}

\newpage{}

\tableref{dq05} below gives a more detailed explanation of {\bf QF-05}. It explains the importance of being comfortable having and using power in the context of group leaders.

\begin{table}[H]
	\caption{Detailed Breakdown of QF-05}\tablelabel{dq05}
	\begin{tabu} to \textwidth {>{\it}l X}
		\toprule
		Qualification Identifier & {\bf QF-05}\\
		Category & Leadership \\
		Phrasing of Qualification & How comfortable are in the seat of power? \\
		Scale for Responses &
		\begin{minipage}[t]{\linewidth}
			\begin{enumerate}
				\item[1.] I am very uncomfortable with power.
				\item[2.] I dislike having power.
				\item[3.] I can handle it if I have to.
				\item[4.] I'm pretty happy taking charge of situations.
				\item[5.] I am very comfortable with having power.
			\end{enumerate}
		\end{minipage}\\
		Reasoning & Groups tend to work better when there is a single or group of people with control. Being comfortable with having and using the power makes for a good quality in a leader figure.\\
		References & Gachter\cite{gachter}\\
		\toprule
	\end{tabu}
\end{table}

\tableref{dq06} below gives a more detailed explanation of {\bf QF-06}. It explains the importance of organization in a group leader.

\begin{table}[H]
	\caption{Detailed Breakdown of QF-06}\tablelabel{dq06}
	\begin{tabu} to \textwidth {>{\it}l X}
		\toprule
		Qualification Identifier & {\bf QF-06}\\
		Category & Leadership \\
		Phrasing of Qualification & Rate your level of organization.\\
		Scale for Responses &
		\begin{minipage}[t]{\linewidth}
			\begin{enumerate}
				\item[1.] Extremely unorganized.
				\item[2.] Unorganized.
				\item[3.] I have a pretty good idea of where everything is.
				\item[4.] Organized.
				\item[5.] Extremely well organized.
			\end{enumerate}
		\end{minipage}\\
		Reasoning & If a person is very disorganized, how can they be trusted with the management of an entire team? Leading a group means keeping up to date with everything that's going on, as well as dealing with issues such as deadlines.\\
		References & Gachter\cite{gachter}\\
		\toprule
	\end{tabu}
\end{table}

\newpage{}

\tableref{dq07} below gives a more detailed explanation of {\bf QF-07}. It explains the importance of having a leader that is comfortable with making huge-impacting decisions.

\begin{table}[H]
	\caption{Detailed Breakdown of QF-07}\tablelabel{dq07}
	\begin{tabu} to \textwidth {>{\it}l X}
		\toprule
		Qualification Identifier & {\bf QF-07}\\
		Category & Leadership \\
		Phrasing of Qualification & Rate your comfortability in making big, important decisions. \\
		Scale for Responses &
		\begin{minipage}[t]{\linewidth}
			\begin{enumerate}
				\item[1.] Very uncomfortable.
				\item[2.] Uncomfortable.
				\item[3.] I don't really mind making decisions if I have to.
				\item[4.] I am pretty comfortable with making tough choices.
				\item[5.] I am very comfortable making tough decisions.
			\end{enumerate}
		\end{minipage}\\
		Reasoning & A leader is always the responsible person in the case that something went wrong. That being said, a leader is also someone that must be very comfortable making large, important decisions that could affect the entire group.\\
		References & Gachter\cite{gachter}, Podsakoff\cite{podsakoff}, Bass\cite{bass} \\
		\toprule
	\end{tabu}
\end{table}

\tableref{dq08} below gives a more detailed explanation of {\bf QF-08}. It explains how important it is for group leaders to be comfortable and willing to step up to the plate and take initiatives.

\begin{table}[H]
	\caption{Detailed Breakdown of QF-08}\tablelabel{dq08}
	\begin{tabu} to \textwidth {>{\it}l X}
		\toprule
		Qualification Identifier & {\bf QF-08}\\
		Category & Leadership \\
		Phrasing of Qualification & How willing are you to step up to the plate and take the initiative?\\
		Scale for Responses &
		\begin{minipage}[t]{\linewidth}
			\begin{enumerate}
				\item[1.] I will not step up and take the initiative.
				\item[2.] I dislike taking charge.
				\item[3.] If no one else wants to, I have no problem taking charge.
				\item[4.] I like stepping up and taking charge.
				\item[5.] I love stepping up and taking charge.
			\end{enumerate}
		\end{minipage}\\
		Reasoning & When one thinks of a stereotypically good leader, charisma is one of the first qualities that come to mind. Good leaders know how to include and incorporate every member of the group in discussions and decisions.\\
		References & Gachter\cite{gachter}, Podsakoff\cite{podsakoff}, Rush\cite{rush}, Bass\cite{bass}\\
		\toprule
	\end{tabu}
\end{table}

\newpage{}

\tableref{dq09} below gives a more detailed explanation of {\bf QF-09}. It explains the importance of having team members with similar work ethics.

\begin{table}[H]
	\caption{Detailed Breakdown of QF-09}\tablelabel{dq09}
	\begin{tabu} to \textwidth {>{\it}l X}
		\toprule
		Qualification Identifier & {\bf QF-09}\\
		Category & Personality \\
		Phrasing of Qualification & Rate your tendency to procrastinate to the last minute on work. \\
		Scale for Responses &
		\begin{minipage}[t]{\linewidth}
			\begin{enumerate}
				\item[1.] I always procrastinate to the last minute.
				\item[2.] I have a tendency to procrastinate.
				\item[3.] Sometimes it happens, sometimes I start early. Depends on my mood.
				\item[4.] I prefer starting earlier rather than later.
				\item[5.] I never procrastinate.
			\end{enumerate}
		\end{minipage}\\
		Reasoning & Like people tend to get along. One of the major causes of contention in groups is work ethic - some people eagerly work early on and finish early as well, whereas  others prefer waiting until the last minute. Generally speaking, combining the two is a terrible idea.\\
		References & Morgeson\cite{morgeson}, Wong\cite{wong}\\
		\toprule
	\end{tabu}
\end{table}

\tableref{dq10} below gives a more detailed explanation of {\bf QF-10}. It explains how sociability is also a factor to consider when building a team.

\begin{table}[H]
	\caption{Detailed Breakdown of QF-10}\tablelabel{dq10}
	\begin{tabu} to \textwidth {>{\it}l X}
		\toprule
		Qualification Identifier & {\bf QF-10}\\
		Category & Personality \\
		Phrasing of Qualification & Rate your tendency to start/begin a discussion with relative strangers. \\
		Scale for Responses &
		\begin{minipage}[t]{\linewidth}
			\begin{enumerate}
				\item[1.] I will never start talking with relative strangers.
				\item[2.] I feel uneasy around strangers.
				\item[3.] If they look approachable I might say hello.
				\item[4.] I enjoy speaking with strangers.
				\item[5.] I will always start talking with random strangers.
			\end{enumerate}
		\end{minipage}\\
		Reasoning & Sociability is an important personality factor to consider when making teams because if a team consists entirely of social people except one, the odd member may feel dejected and ignored. On the otherhand, a team of anti-social people except one would have similar effects.\\
		References & Morgeson\cite{morgeson}\\
		\toprule
	\end{tabu}
\end{table}

\newpage{}

\tableref{dq11} below gives a more detailed explanation of {\bf QF-11}. It explains how important working style is when thinking of group cohesion.

\begin{table}[H]
	\caption{Detailed Breakdown of QF-11}\tablelabel{dq11}
	\begin{tabu} to \textwidth {>{\it}l X}
		\toprule
		Qualification Identifier & {\bf QF-11}\\
		Category & Personality \\
		Phrasing of Qualification & When working on large scale projects, what is your preferred way of working? \\
		Scale for Responses &
		\begin{minipage}[t]{\linewidth}
			\begin{enumerate}
				\item[1.] I start everything at once and distribute my time across everything evenly.
				\item[2.] I try to do a bit of everything, but I tend to get excited/focused on one part and finish it completely.
				\item[3.] I try to do things one a time, but usually end up starting multiple things anyways.
				\item[4.] I focus on one thing at a time until it's finished.
			\end{enumerate}
		\end{minipage}\\
		Reasoning & Work habits are an aspect of a person's personality that may result in intra-group contention - if one member is trying to finish things one-by-one whereas the other is just starting everything at once, work may overlap and get difficult to correctly organize.\\
		References & Morgeson\cite{morgeson}, Wong\cite{wong}\\
		\toprule
	\end{tabu}
\end{table}

\tableref{dq12} below gives a more detailed explanation of {\bf QF-12}. It explains the importance of similar levels of punctuality in group members.

\begin{table}[H]
	\caption{Detailed Breakdown of QF-12}\tablelabel{dq12}
	\begin{tabu} to \textwidth {>{\it}l X}
		\toprule
		Qualification Identifier & {\bf QF-12}\\
		Category & Personality \\
		Phrasing of Qualification & If you have a meeting tomorrow, what time would you usually arrive? \\
		Scale for Responses &
		\begin{minipage}[t]{\linewidth}
			\begin{enumerate}
				\item[1.] I don't usually show up to meetings.
				\item[2.] Considerably late.
				\item[3.] About on time - perhaps a few minutes either way.
				\item[4.] Several minutes early.
				\item[5.] More than fifteen minutes early.
			\end{enumerate}
		\end{minipage}\\
		Reasoning & Punctuality, while not limited to groups, is an issue that can lead to heavy annoyance. Always having that one guy that arrives an hour late can lead to internal strife.\\
		References & Tjosvold\cite{tjosvold}\\
		\toprule
	\end{tabu}
\end{table}

\newpage{}

\tableref{dq13} below gives a more detailed explanation of {\bf QF-13}. It explains the importance common working hours for group members.

\begin{table}[H]
	\caption{Detailed Breakdown of QF-13}\tablelabel{dq13}
	\begin{tabu} to \textwidth {>{\it}l X}
		\toprule
		Qualification Identifier & {\bf QF-13}\\
		Category & Availability \\
		Phrasing of Qualification & During a typical day, when is your preffered working time? \\
		Scale for Responses &
		\begin{minipage}[t]{\linewidth}
			\begin{enumerate}
				\item[1.] Very late at night.
				\item[2.] During the evening.
				\item[3.] During the afternoon.
				\item[4.] During the morning.
				\item[5.] Very early in the morning.
			\end{enumerate}
		\end{minipage}\\
		Reasoning & The worst thing when trying to coordinate working on group projects is having that one guy that can only work at obscenely early or late hours or the day. Having a group with common working times is vitally important.\\
		References & Tjosvold\cite{tjosvold}\\
		\toprule
	\end{tabu}
\end{table}

\tableref{dq14} below gives a more detailed explanation of {\bf QF-14}. It explains the importance of common working days for group members.

\begin{table}[H]
	\caption{Detailed Breakdown of QF-14}\tablelabel{dq14}
	\begin{tabu} to \textwidth {>{\it}l X}
		\toprule
		Qualification Identifier & {\bf QF-14}\\
		Category & Availability \\
		Phrasing of Qualification & Based on your current schedule, which days are you usually available? \\
		Scale for Responses &
		\begin{minipage}[t]{\linewidth}
			\begin{enumerate}
				\item[1.] Never.
				\item[2.] Weekends only.
				\item[3.] Some weekdays only.
				\item[4.] Any day of the week, but not the weekends.
				\item[5.] Every day is a good day.
			\end{enumerate}
		\end{minipage}\\
		Reasoning & Having a general idea of which days one is available can be an important factor when considering teams - an otherwise okay team might suffer if half the members can only meet on Mondays whereas the other half can only meet on Sundays.\\
		References & Tjosvold\cite{tjosvold}\\
		\toprule
	\end{tabu}
\end{table}

\newpage{}

\tableref{dq15} below gives a more detailed explanation of {\bf QF-15}. It explains the importance of having a similar method of communication across the entire group.

\begin{table}[H]
	\caption{Detailed Breakdown of QF-15}\tablelabel{dq15}
	\begin{tabu} to \textwidth {>{\it}l X}
		\toprule
		Qualification Identifier & {\bf QF-15}\\
		Category & Availability \\
		Phrasing of Qualification & Based on how often you check, what is your preferred mode of communication? \\
		Scale for Responses &
		\begin{minipage}[t]{\linewidth}
			\begin{enumerate}
				\item[1.] Phone calls.
				\item[2.] Text messages.
				\item[3.] E-mails.
				\item[4.] Forums.
			\end{enumerate}
		\end{minipage}\\
		Reasoning & It's nice to have a common mode of communication within groups - it allows simple and consistent communication between any and all group members. It's always annoying to have that one guy that only answers his phone and everyone else works through email.\\
		References & Tjosvold\cite{tjosvold}\\
		\toprule
	\end{tabu}
\end{table}

\subsection{Rules}
\subsubsection{Overview of Rules}

The following table, \tableref{rc}, outlines and describes all the different possible types of rules that we will be using. The categories below are categories of rules, not the rules themselves.

\begin{table}[H]
	\caption{Rule Categories/Classes}\tablelabel{rc}
	\begin{tabu} to \textwidth {>{\bf}l >{\it}l X}
	    \tableheader{}Identifier & Name & Description\\
		RC-01 & 4-point Scale & Rules in this category apply to Qualifications that use a 4-point Scale.\\
		RC-02 & 5-point Scale & Rules in this category apply to Qualifications that use a 5-point Scale.\\
		RC-03 & Percent-Match & Rules in this category handle converting responses to a percentage value.\\
		RC-04 & Percent-Total & Rules in this category convert a collection of response percentage values to a Total Match Percent.\\
		RC-05 & Personal & Rules in this category handle Personal Values. \\
		RC-06 & Desired & Rules in this category handle Desired Values.\\
	\end{tabu}
\end{table}

\begin{center}
\begin{tabu} to \textwidth {>{\bf}l >{\it}l X}
	    \tableheader{}Identifier & Name & Description\\
		RC-07 & Part 1 & Rules in this category are a part of the individual Percent-Match component of the algorithm.\\
		RC-08 & Part 2 & Rules in this category are a part of the Group-Matching component of the algorithm.\\
		RC-09 & Special Cases & Rules in this category deal with special cases in the algorithm.\\
\end{tabu}
\end{center}

\tableref{or} is a table that contains all the different rules used to manipulate the raw results of the qualifications into values that we can actually make use of. Alongside the unique ID for each, you will also notice that there is a column for the categories that the rule in question is a part of. Frequently, rules are part of several different categories. There are two major sections within the different types of rules: rules that apply to responses and generates the percent-compatibility with another specified individual ({\bf R-01} to {\bf R-13}), and rules that build the teams based on those percent-compatibilities ({\bf R-14} to {\bf R-24}).

\begin{table}[H]
	\caption{Overview of Rules}\tablelabel{or}
	\begin{tabu} to \textwidth {>{\bf}l >{\it}p{2.75cm} X}
	    \tableheader{}Identifier & Categories & Description\\
		R-01 & RC-01 & Qualifications use percent increments of 25\%.\\
		R-02 & RC-02 & Qualifications use percent increments of 20\%.\\
		R-03 & RC-01, RC-03, RC-05, RC-06, RC-07 & A Response matching the desired Response is given a Match Percent of 100\%.\\
		R-04 & RC-01, RC-03, RC-05, RC-06, RC-07 & A Response that is directly to the left or to the right of the desired Response is given a Match Percent of 75\%\\
		R-05 & RC-01, RC-03, RC-05, RC-06, RC-07 & A Response that is two places to the left or to the right of the desired Response is given a Match Percent of 50\%\\
		R-06 & RC-01, RC-03, RC-05, RC-06, RC-07 & A Response that is three places to the left or to the right of the desired Response is given a Match Percent of 25\%\\
		R-07 & RC-02, RC-03, RC-05, RC-06, RC-07 & A Response matching the desired Response is given a Match Percent of 100\%.\\
		R-08 & RC-02, RC-03, RC-05, RC-06, RC-07 & A Response that is directly to the left or to the right of the desired Response is given a Match Percent of 80\%\\
		R-09 & RC-02, RC-03, RC-05, RC-06, RC-07 & A Response that is two places to the left or to the right of the desired Response is given a Match Percent of 60\%\\
	\end{tabu}
\end{table}

\begin{center}
    \begin{tabu} to \textwidth {>{\bf}l >{\it}p{2.75cm} X}
	    \tableheader{}Identifier & Categories & Description\\
		
		R-10 & RC-02, RC-03, RC-05, RC-06, RC-07 & A Response that is three places to the left or to the right of the desired Response is given a Match Percent of 40\%\\
		R-11 & RC-02, RC-03, RC-05, RC-06, RC-07 & A Response that is four places to the left or to the right of the desired Response is given a Match Percent of 20\%\\
		R-12 & RC-04, R-07 & A Student's degree of match to another Student is calculated via averaging all the Match Percents found. \\
		R-13 & RC-07 & The final Match Percentage between two Students is calculated by averaging each Student's Match Percent with each other's.\\
		R-14 & RC-08 & A Student is compared to every other Student using R-03 to R-13 and the final Match Percentages are stored.\\
		R-15 & RC-08 & If R-14 is run successfully, the two best matches for every Student is kept.\\
		R-16 & RC-08 & If R-15 is run successfully, out of all Students, the worst match percent is found.\\
		R-17 & RC-08 & If R-16 is run successfully, the Student that has the worst match is then rematched with their best match using R-03 to R-13.\\
		R-18 & RC-08 & If R-17 is run successfully, then the two matched Students are put into a group together.\\
		R-19 & RC-08 & R-14 is repeated until all groups are within the min/max range or until only one group is left that is not at the max group size (where said group's size is below the min).\\
		R-20 & RC-08, RC-09 & After R-19 is completed, the largest group size is compared to the lowest group size, and if the difference is greater than 1 and not within the min/max range requested, then initiate R-21.\\
		R-21 & RC-08, RC-09 & Every member in the larger groups is compared to every member of the smaller group using R-03 to R-13. \\
		R-22 & RC-08, RC-09 & After R-21, the member of the larger group with the highest match to the members of the smallest group that is not yet at the min size is then moved to the smaller group.\\
		R-23 & RC-08, RC-09 & R-20 is repeated until either all groups are within the min/max range specified, or until all groups are within $\pm$ 1 of each other (in the situation where the first condition is impossible to attain).\\
		R-24 & RC-08, RC-09 & A Student that is not part of a group of 2 or more is considered to be in a group of size 1.
    \end{tabu}
\end{center}

\newpage{}

\subsubsection{Detailed Breakdown of Rules}
\subsubsection*{Individual Match-Percent Rules}
The following figure, {\bf Figure 1},  is a visual representation of the flow of the rule application with regards to the first major section of rules (percent-compatibility between individuals). The process can be broken down into four segments:
\begin{enumerate}
	\item[1.] Students complete their profiles.
	\item[2.] Rules are cross-applied to each Student's profiles, where their desired values get compared to the personal values of other Student's and vice-versa. {\bf R-03} to {\bf R-11}
	\item[3.] The resulting percentage values for each comparison is averaged within each person. {\bf R-12}
	\item[4.] The average percentage of a person's match to another is averaged with the average percentage of that other person's match to the original person. {\bf R-13}
\end{enumerate}

\begin{figure}[H]
	\caption{Flow of Rule Application Between Individuals}
	\begin{center}
		\includegraphics[scale=0.55]{imgs/algorithm-design-figure-1.png}
	\end{center}
\end{figure}
\subsubsection*{Group Making Rules}
The second major section of rules deals with the creation of teams using the first section as a basis for matching. The process can be broken down into several segments:
\vspace{1em}
\begin{enumerate}
	\item[1.] A group of Students have filled out their respective profiles and the Administrator begins running the algorithm. The following figure, {\bf Figure 2}, shows five students that have successfully completed their profiles and are ready to be sorted. 
\begin{figure}[H]
	\caption{Initiation Step}
	\begin{center}
		\includegraphics[scale=0.78]{imgs/algorithm-design-figure2.png}
	\end{center}
\end{figure}
	\item[2.] Every Student is then compared to every other Student that is a part of the current project using the first section's percent-match algorithm. {\bf Figure 3} shows {\it Stu1} being compared to every other Student and their match-percents are calculated. {\bf R-14}
\begin{figure}[H]
	\caption{One-To-Many Percent Comparison}
	\begin{center}
		\includegraphics[scale=0.68]{imgs/algorithm-design-figure3.png}
	\end{center}
\end{figure}
\newpage{}
	\item[3.] Of all the comparisons, the algorithm will keep the best two matches for each Student. {\bf Figure 4} focuses on the two highest values. {\bf R-15}
\begin{figure}[H]
	\caption{Best-Two-of-Many Decision}
	\begin{center}
		\includegraphics[scale=0.68]{imgs/algorithm-design-figure4.png}
	\end{center}
\end{figure}
	\item[4.] This process, as mentioned earlier, gets repeated with every Student such that every Student has their best two matches. The following figure, {\bf Figure 5}, shows the result of steps 2-3 applied on every Student. {\bf R-16}
\begin{figure}[H]
	\caption{Best-Two-of-Many For All Students}
	\begin{center}
		\includegraphics[scale=0.68]{imgs/algorithm-design-figure5.png}
	\end{center}
\end{figure}
\newpage{}
	\item[5.] The algorithm then selects the match amongst all Students that is the lowest match percent. In {\bf Figure 6}, the lowest match is showed via a bold, blue arrow. Keep in mind here that the lowest match is for {\it Stu4}. {\bf R-17}
\begin{figure}[H]
	\caption{Lowest Match Selection}
	\begin{center}
		\includegraphics[scale=0.68]{imgs/algorithm-design-figure6.png}
	\end{center}
\end{figure}
	\item[6.] The algorithm then finds his best match using the first section's algorithm. {\bf Figure 7} shows {\it Stu4}'s actual best match. {\bf R-17}
\begin{figure}[H]
	\caption{Highest Match Selection}
	\begin{center}
		\includegraphics[scale=0.68]{imgs/algorithm-design-figure7.png}
	\end{center}
\end{figure}
	\item[7.] These two matched individuals are then grouped together. {\bf Figure 8} shows that the these two individuals are put into a group together, portrayed here as being in an ellipse together.{\bf R-18}
\begin{figure}[H]
	\caption{Grouping Matches Together}
	\begin{center}
		\includegraphics[scale=0.68]{imgs/algorithm-design-figure8.png}
	\end{center}
\end{figure}
\newpage{}
	\item[8.] This process is then repeated over and over again until all groups are within the min/max range, or until every group but 1 is at the max group size and the last group falls below the min size. The following figure, {\bf Figure 9}, shows the eventual final result of the algorithm after it has run successfully. {\bf R-19}
\begin{figure}[H]
	\caption{Final Result}
	\begin{center}
		\includegraphics[scale=0.68]{imgs/algorithm-design-figure9.png}
	\end{center}
\end{figure}

\end{enumerate}
\subsubsection{Special Cases}
\subsubsection*{Case 1: A Group is Being Compared to a Single Person/Group}
In the circumstance that an already formed group is being compared to either another formed group or a single individual, that group's match value is equivalent to the average match value of every person in that group with the Student or group that it is being compared to. 
\subsubsection*{Case 2: Leftover Too-Small Groups}
In the event that the final groups sizes are off by more than $\pm$ 1 then the balancing portion of the algorithm begins.
\vspace{1em}
\begin{enumerate}
	\item[1.] Groups that don't fall in the min/max range have been formed. The following figure, {\bf Figure 10} shows two groups, wherein the first group consists of a single Student, whereas the second group consists of four students, and where the min is 3 and the max is 4. {\bf R-20}
\begin{figure}[H]
	\caption{Initial Step For Leftover Small Groups}
	\begin{center}
		\includegraphics[scale=0.78]{imgs/algorithm-design-figure10.png}
	\end{center}
\end{figure}
\newpage{}
	\item[2.] In all groups where the number of people in it is less than the max group size, each member gets compared to every member of the group that is too small using the match-percent algorithm outlined in the previous section. The below figure, {\bf Figure 11}, shows how every individual in a larger group is matched with each individual in the smaller group, where {\it Stu1} is in a group of size 1. {\bf R-21}
\begin{figure}[H]
	\caption{Many-to-One Percent Comparison}
	\begin{center}
		\includegraphics[scale=0.68]{imgs/algorithm-design-figure11.png}
	\end{center}
\end{figure}
	\item[3.] The Student from the smaller group that has the highest match-percent with the larger group is then moved over to the larger group. In {\bf Figure 12}, we see that {\it Stu3} is the best match for {\it Stu1}.
\begin{figure}[H]
	\caption{Selecting Highest Percent-Match}
	\begin{center}
		\includegraphics[scale=0.68]{imgs/algorithm-design-figure12.png}
	\end{center}
\end{figure}
	\item[4.] The process is repeated until all groups are within the max and min range. {\bf Figure 13} shows how {\it Stu1} then gets moved over into the larger group. {\bf R-23}
\begin{figure}[H]
	\caption{Rebalancing of Groups}
	\begin{center}
		\includegraphics[scale=0.68]{imgs/algorithm-design-figure14.png}
	\end{center}
\end{figure}
\end{enumerate}
\newpage{}
\subsubsection{Case 3: Insufficient Students to Form Desired Group Sizes}

In the event that the amount of Students is insufficient, the algorithm will still run and sort the pool of Students into groups of approximately the same size ($\pm$ 1). The algorithm however will also flag to the Administrator the groups may be smaller than intended. This algorithm component works similarly to the previous Case's, except that instead of moving the member from the smallest group to the larger group, a member from the larger group gets moved to the smaller group. {\bf R-22}

\renewcommand\refname{\vskip -1cm}
\section{References}

\begin{thebibliography}{16}
\bibitem{bass}
Bass, Bernard M.
{\it Leadership: Good, Better, Best},
The Free Press, 1985.

\bibitem{gachter}
Gachter, Simon et al.
{\it Who Makes a Good Leader? Cooperativeness, Optimism, and Leading-by-Example},
Economic Inquiry, 2010.

\bibitem{claurend1}
Laurendeau, Christine.
{\it COMP 2401 -- Introduction to Systems Programming: Course Outline},
Carleton University, Ottawa, Ontario, 2014.

\bibitem{claurend2}
Laurendeau, Christine.
{\it COMP 2404 -- Introduction to Software Engineering: Course Outline},
Carleton University, Ottawa, Ontario, 2014.

\bibitem{morgeson}
Morgeson, Frederick P. et al.
{\it Selecting Individuals in Team Settings: The Importance of Social Skills, Personality Characteristics, and Teamwork Knowledge},
Personnel Psychology, East Lansing, Michigan, 2005.

\bibitem{podsakoff}
Podsakoff, Philip M.
{\it Transformational leader behaviors and their effects on followers' trust in leader, satisfaction, and organizational citizenship behaviors},
The Leadership Quarterly, Bloomington, Indiana, 1990.

\bibitem{rush}
Rush, Michael C. and Russel, Joyce E. A.
{\it Leader Prototypes and Prototype-Contingent Consensus in Leader Behavior Descriptions},
Journal of Experimental Social Psychology, Tennessee, 1986.

\bibitem{tjosvold}
Tjosvold, Dean et al.
{\it Conflict values and team relationships: conflict's contribution to team effectiveness and citizenship in China},
Journal of Organizational Behavior, Hong Kong, China, 2001.

\bibitem{wong}
Wong, Sze-sze and Burton, Richard M.
{\it Virtual Teams: What are their Characteristics, and Impact on Team Performance?},
Kluwer Academic Publishers, Durham, North Carolina, 2001.

\end{thebibliography}

\end{document}
